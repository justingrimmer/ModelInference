\documentclass{beamer}

%\usepackage[table]{xcolor}
\mode<presentation> {
  \usetheme{Boadilla}
%  \usetheme{Pittsburgh}
%\usefonttheme[2]{sans}
\renewcommand{\familydefault}{cmss}
%\usepackage{lmodern}
%\usepackage[T1]{fontenc}
%\usepackage{palatino}
%\usepackage{cmbright}
  \setbeamercovered{transparent}
\useinnertheme{rectangles}
}
%\usepackage{normalem}{ulem}
%\usepackage{colortbl, textcomp}
\setbeamercolor{normal text}{fg=black}
\setbeamercolor{structure}{fg= black}
\definecolor{trial}{cmyk}{1,0,0, 0}
\definecolor{trial2}{cmyk}{0.00,0,1, 0}
\definecolor{darkgreen}{rgb}{0,.4, 0.1}
\newtheorem{prop}{Proposition}
\newtheorem{defn}{Definition}
\newtheorem{cor}{Corollary}
\newtheorem{lem}{Lemma} 

\usefonttheme{professionalfonts} % using non standard fonts for beamer
\usefonttheme{serif}

\beamertemplatesolidbackgroundcolor{white}  \setbeamercolor{alerted
text}{fg=red}
\usepackage{multirow}

\setbeamertemplate{caption}[numbered]\newcounter{mylastframe}

%\begin{beamerboxesrounded}[upper=uppercol,lower=lowercol,shadow=true]{Block}
%$A = B$.
%\end{beamerboxesrounded}}
\renewcommand{\familydefault}{cmss}
\usepackage[all]{xy}
\usepackage{tikz}
\usetikzlibrary{arrows}
%\usepackage{palatino}
%\usepackage{eulervm}
\usecolortheme{lily}
%\usepackage[latin1]{inputenc}

\title[Methodology III] % (optional, nur bei langen Titeln nötig)
{Political Methodology III: Model Based Inference}

\author{Justin Grimmer}
\institute[Stanford University]{Associate Professor\\Department of Political Science \\  Stanford University}
\vspace{0.3in}

\date{April 17th, 2017}

\begin{document}
\begin{frame}
\titlepage
\end{frame}



\begin{frame}
\frametitle{Most Important Problems}
Effects of presidents going public\\
\begin{itemize}
\item[-] $Y_{i} $ = 1 if respondent identifies MIP as topic of president's speech 
\item[-] $\boldsymbol{X}_{i}  = $ (1, Treatment, Republican)
\end{itemize}

\end{frame}


\begin{frame}
\frametitle{Review: Probit}

\begin{eqnarray}
\underbrace{Y_{i}}_{\text{Random}} & \sim & \text{Bernoulli}(\pi_{i}) \nonumber \pause \\
\invisible<1>{\pi_{i} & = & \Phi(\boldsymbol{X}_{i}^{'}\underbrace{\boldsymbol{\beta}}_{\text{Infer (Fixed)}}) \nonumber } \pause 
\end{eqnarray}


\begin{eqnarray}
\invisible<1-2>{L(\boldsymbol{\beta} | \boldsymbol{X}, \boldsymbol{Y} ) & = & f(\boldsymbol{Y}| \boldsymbol{\beta}, \boldsymbol{X}) \nonumber \\} \pause 
\invisible<1-3>{& =& \prod_{i=1}^{N} \Phi(\boldsymbol{X}_{i}^{'} \boldsymbol{\beta})^{Y_{i}} (1- \Phi(\boldsymbol{X}_{i}^{'}\boldsymbol{\beta} ))^{1- Y_{i}} \nonumber \\} \pause 
\invisible<1-4>{\log L(\boldsymbol{\beta} | \boldsymbol{X}, \boldsymbol{Y} ) &= & \sum_{i=1}^{N} \left(Y_{i} \log \Phi(\boldsymbol{X}_{i}^{'} \boldsymbol{\beta}) + (1- Y_{i} )\log (1- \Phi(\boldsymbol{X}_{i}^{'} \boldsymbol{\beta}) ) \right) \nonumber \\} \pause 
\invisible<1-5>{\boldsymbol{\beta}^{*} & = & \boldsymbol{\beta}^{*}(\boldsymbol{X}, \boldsymbol{Y}) =  \text{Maximum Likelihood Estimator} \nonumber \\ } \pause 
\invisible<1-6>{\widehat{\boldsymbol{\beta}  } & = & \boldsymbol{\beta}^{*}(\boldsymbol{X}, \boldsymbol{y} )} \nonumber 
\end{eqnarray}


\end{frame}



\begin{frame}
\frametitle{Review: Probit}

\begin{eqnarray}
\boldsymbol{\beta}^{*} & \rightarrow^{D}&  \text{Multivariate Normal}\left(\boldsymbol{\beta}, I_{N} (\widehat{\boldsymbol{\beta }})^{-1} \right) \nonumber \pause  \\
\invisible<1>{\boldsymbol{\beta}^{*} & \rightarrow^{D}&  \text{Multivariate Normal}\left(\boldsymbol{\beta}, -E[\text{Hessian}_{i}(\widehat{\boldsymbol{\beta}})]^{-1} \right) \nonumber } \pause 
\end{eqnarray}

\begin{eqnarray}
\invisible<1-2>{I_{N} (\widehat{\boldsymbol{\beta }})^{-1} & = & 
\begin{pmatrix}
\text{var}(\beta_{0})&  \text{cov}(\beta_{0},\beta_{1}) & \hdots & \text{cov}(\beta_{0}, \beta_{K}) \\
\text{cov}(\beta_{1}, \beta_{0}) & \text{var}(\beta_{1}) & \hdots & \text{cov}(\beta_{1}, \beta_{K}) \\ 
\vdots & \vdots & \ddots & \vdots \\
\text{cov}(\beta_{K}, \beta_{0}) & \text{cov}(\beta_{K}, \beta_{1}) & \hdots & \text{var}(\beta_{K})
 \end{pmatrix} \nonumber }
\end{eqnarray}




\end{frame}



\begin{frame}
\frametitle{Review Probit}

\pause 

\invisible<1>{For a probit regression coefficient: } \pause 
\begin{eqnarray}
\invisible<1-2>{\text{CI}_{95} & = & [\widehat{\beta}_{k} -  1.96 \times \sqrt{\text{var}(\beta_{k})}, \widehat{\beta}_{k} + 1.96 \times \sqrt{\text{var}(\beta_{k})}]  \nonumber \\} \pause 
\invisible<1-3>{\text{CI}_{1-\alpha} & = & [\widehat{\beta}_{k} -  z_{\alpha/2} \times \sqrt{\text{var}(\beta_{k})}, \widehat{\beta}_{k} + z_{\alpha/2} \times \sqrt{\text{var}(\beta_{k})}] \nonumber } \pause 
\end{eqnarray}
\invisible<1-4>{Recall: If $Z\sim \text{Normal}(0,1)$, $z_{\alpha/2}$ is the number such that 

\begin{eqnarray}
P(|Z\geq z_{\alpha/2}|) & = & \alpha  \nonumber 
\end{eqnarray}}

\end{frame}


\begin{frame}
\frametitle{Review Probit}

\begin{semiverbatim}

>colnames(mips)


[1] "mip"   "treat" "gop"  



>reg1<- glm(mip$\sim$treat + gop, family = binomial(link='probit'), data = as.data.frame(mips))


> reg1\$coef

 (Intercept)        treat          gop 

-0.259173644 -0.009598938 -0.198126517 



> sqrt(diag(vcov(reg1)))

(Intercept)       treat         gop 

 0.03297588  0.05268254  0.05115777 

\end{semiverbatim}



\end{frame}



\begin{frame}
\frametitle{Review Probit}

95-percent confidence interval:
\begin{semiverbatim}
> ci\_treat<- c(reg1\$coef[2] - 1.96\*sqrt(diag(vcov(reg1)))[2], reg1\$coef[2] + 1.96\*sqrt(diag(vcov(reg1)))[2])


> ci\_treat




-0.11285672  0.09365884 

\end{semiverbatim}





\end{frame}




\begin{frame}
\frametitle{Review Probit}

Quantities of interest: \pause 
\begin{eqnarray}
\invisible<1>{E[Y_{i}| \boldsymbol{X}_{i}] & = & \Phi(\boldsymbol{X}^{'}_{i} \boldsymbol{\beta}) \nonumber }\pause  \\
\invisible<1-2>{E[Y_{i}| X_{i1} = 1, \boldsymbol{X}_{i}] - E[Y_{i}| \tilde{X}_{i1} = 0, \tilde{\boldsymbol{X}}_{i}] &= & \Phi(\boldsymbol{X}^{'}_{i} \boldsymbol{\beta}) - \Phi(\tilde{\boldsymbol{X}}^{'}_{i} \boldsymbol{\beta}) \nonumber}\pause  \\
\invisible<1-3>{\frac{ \partial E[Y_{i}| X_{i1} = 1, \boldsymbol{X}_{i}]}{\partial X_{i1} } & = & \phi(\boldsymbol{X}^{'}_{i} \boldsymbol{\beta}) \beta_{1} \nonumber }
\end{eqnarray}



\end{frame}


\begin{frame}

\begin{semiverbatim}

> X\_synth<- c(1, 0, 1)


> y.tilde<- reg1\$coef\%*\%X\_synth


> y.prob<- pnorm(y.tilde)


> y.prob


          
[1,] 0.3237277


\end{semiverbatim}



\end{frame}




\begin{frame}
\frametitle{Uncertainty for Maximum Likelihood Estimation}

\pause 
\begin{itemize}
\invisible<1>{\item[1)] Coefficients: \alert{Asymptotic result} $\leadsto$ YMMV in small samples} \pause 
\invisible<1-2>{\item[2)] And even if asymptotic distribution holds:} \pause 
\begin{itemize}
\invisible<1-3>{\item[1)] Difficult substantive interpretation of coefficients} \pause 
\invisible<1-4>{\item[2)] Uncertainty for \alert{functions} of coefficients} 
\end{itemize}



\end{itemize}



\end{frame}



\begin{frame}
\frametitle{Inference Three Ways}

How do obtain uncertainty estimates for Quantities of Interest?


\begin{itemize}
\item[1)] Bootstrap $\leadsto$ (no asymptotics, simulation)
\item[2)] Delta Method $\leadsto$ (asymptotic normality, analytic)
\item[3)] Simulation from Multivariate Normal $\leadsto$ (asymptotic normality, simulation)
\end{itemize}


\end{frame}

\begin{frame}
\frametitle{The Bootstrap (Procedure, Then Justification)}

Suppose we have maximum likelihood estimator $\boldsymbol{\beta}^{*}(\boldsymbol{X}, \boldsymbol{Y})$ for $N$ observations


For each of $M$ simulations $(m = 1,2, \hdots, M)$: \pause 
\begin{itemize}
\invisible<1>{\item[1)] Sample \alert{with replacement} $N$ times: } \pause 
\begin{eqnarray}
\invisible<1-2>{(\tilde{Y}_{1}, \tilde{\boldsymbol{X}}_{1}), (\tilde{Y}_{2}, \tilde{\boldsymbol{X}}_{2}) & \hdots & (\tilde{Y}_{N}, \tilde{\boldsymbol{X}}_{N}) = (\tilde{Y}, \tilde{\boldsymbol{X}}) \nonumber } \pause 
\end{eqnarray}
\invisible<1-3>{\item[2)] Calculate $\boldsymbol{\beta}^{*}(\boldsymbol{X}, \boldsymbol{Y})$ and/or \alert{any other function} of the estimator (expected value, first difference) on bootstrapped sample} \pause 
\begin{eqnarray}
\invisible<1-4>{\boldsymbol{\beta}^{m} & = & \boldsymbol{\beta}^{*}(\tilde{Y}, \tilde{\boldsymbol{X}}) \nonumber\\} \pause 
\invisible<1-5>{h(\boldsymbol{\beta})^{m} & = & h(\boldsymbol{\beta}^{*}(\tilde{Y}, \tilde{\boldsymbol{X}})) \nonumber } \pause 
\end{eqnarray}
\invisible<1-6>{\item[3)] Do following: } \pause 
\begin{eqnarray}
\only<1-8>{\invisible<1-7>{E[\boldsymbol{\beta}^{*}] & \approx &  \sum_{m=1}^{M} \frac{\boldsymbol{\beta}^{m} }{M} \nonumber }}
\only<9>{\text{Var}(\boldsymbol{\beta}^{*}) & \approx & \sum_{m=1}^{M} \frac{\left(\boldsymbol{\beta}^{m} -      (\sum_{m=1}^{M} \frac{\boldsymbol{\beta}^{m} }{M}) \right)^{2}}{M} \nonumber }
\only<10>{E[h(\boldsymbol{\beta}^{*})] & \approx & \sum_{m=1}^{M} \frac{ h(\boldsymbol{\beta})^{m}  }{M} \nonumber }
\end{eqnarray}

\end{itemize}

\end{frame}


\begin{frame}
\frametitle{Example: }

\begin{semiverbatim}
store\_expected<- c()

for(z in 1:1000)\{

\hspace{0.2in}	subset<- sample(1:nrow(mips), nrow(mips), replace=T)

\hspace{0.2in}	use\_mips<- mips[subset,]

\hspace{0.2in}	temp\_reg<- glm(mip$\sim$treat + gop, 

\hspace{0.4in}		family = binomial(link='probit'), 

\hspace{0.4in}		data = as.data.frame(use\_mips))

\hspace{0.2in}	store\_expected[z]<- pnorm(temp\_reg\$coef\%*\%X\_synth)

	\}

> mean(store\_expected) \#\# estimate of expected value

[1] 0.3245165

> sd(store\_expected) \#\# estimate of standard error

[1] 0.0157724
\end{semiverbatim}




\end{frame}


\begin{frame}
\frametitle{The Bootstrap: Justification} 
Two key ideas:
\begin{itemize}
\item[1)] Simulation
\item[2)] Approximation of cumulative distribution function $F$ with empirical distribution function $\hat{F}_{n}$
\end{itemize}


\end{frame}




\begin{frame}
\frametitle{Simulation}

\begin{itemize}
\item[-] Sample $M$ draws,  $Y_{m} \sim F$ 
\item[-] For example $Y_{m} \sim \text{Normal}(\mu, \sigma^2) $
\end{itemize}
\pause 

\begin{eqnarray}
\invisible<1>{\frac{1}{M}\sum_{m=1}^{M}Y_{m} & \longrightarrow^{P} & E[Y] = \int y f(y) dy =  \int y dF(y) \nonumber \\} \pause 
\invisible<1-2>{\frac{1}{M}\sum_{m=1}^{M}h(Y_{m}) & \longrightarrow^{P} &E[h(Y)] = \int h(y)f(y) dy  \nonumber \\} \pause 
\invisible<1-3>{\frac{1}{M}\sum_{m=1}^{M} \left(Y_{m} - \sum_{m=1}^{M} \frac{Y_{m}}{M} \right)^2 & \longrightarrow^{P} & \int (y-\mu)^2 f(y)dy \nonumber } 
\end{eqnarray}

\end{frame}


\begin{frame}
\frametitle{Empirical Distribution Function}


\pause 

\invisible<1>{Univariate version: $X_{i}$ is a scalar
\begin{eqnarray}
\hat{F}_{n}(x) & = & \sum_{i=1}^{N} \frac{I(X_{i} \leq x)  }{ N } \nonumber 
\end{eqnarray}}
\pause 

\invisible<1-2>{Multivariate Version: $\boldsymbol{X}_{i}$ is a vector 
\begin{eqnarray}
\hat{F}_{n}(\boldsymbol{x}) & = & \sum_{i=1}^{N} \frac{I(\boldsymbol{X}_{i} \leq  \boldsymbol{x} ) }{ N} \nonumber 
\end{eqnarray}
}
\pause 
\invisible<1-3>{Empirical density function: every observation has density height (probability) $\frac{1}{N}$.  }



\end{frame}




\begin{frame}
\frametitle{Approximation of Cumulative Distribution Function $F$}

Expected value depends on \alert{$F$}  \pause 
\begin{eqnarray}
\invisible<1>{\text{Mean}_{F}(Y) & = & E[Y] = \int y dF(y) \nonumber \\} \pause 
\invisible<1-2>{\text{Var}_{F}(Y) & = & E[(Y- \mu)^2] = \int (y- \mu)^2 dF(y) \nonumber \\} \pause 
\invisible<1-3>{\text{h}_{F}(Y) &= & E[h(Y)] = \int h(y)dF(y) \nonumber \\} \pause 
\invisible<1-4>{\text{Mean}_{F}(Y) & \approx & \text{Mean}_{\hat{F}}(Y)) = \int y d\hat{F}(y) \nonumber \\} \pause
\invisible<1-5>{\text{Var}_{F}(Y) & \approx & \text{Var}_{\hat{F}}(Y) = \int (y- \mu)^2 d\hat{F}(y) \nonumber \\} \pause
\invisible<1-6>{\text{h}_{F}(Y) & \approx & \text{h}_{\hat{F}}(Y) = \int h(y) d\hat{F}(y) \nonumber } 
\end{eqnarray}


\end{frame}


\begin{frame}
\frametitle{How to Take Expected Values Using $\hat{F}_{N}$?}
\pause \invisible<1>{Simulation!} \\ \pause 

\invisible<1-2>{Bootstrap is a double approximation:} \pause 
\begin{eqnarray}
\invisible<1-3>{\text{h}_{F}(Y) }\pause\invisible<1-4>{& \approx & \text{h}_{\hat{F}}(Y)}\pause\invisible<1-5>{ \approx \sum_{m=1}^{M} \frac{ h(Y_{m} )  }{ M } } \pause \nonumber \\
\invisible<1-6>{h_{F}(\boldsymbol{\beta}^{*}) & \approx & h_{\hat{F}}(\boldsymbol{\beta}^{*}) \approx \sum_{m=1}^{M} \frac{h(\boldsymbol{\beta}^{*}_{m})}{M} \nonumber }
\end{eqnarray}

\end{frame}






\begin{frame}
\frametitle{Percentile Bootstrap Confidence Intervals}

\begin{itemize}
\item[-] Suppose we have $M$ Bootstrap iterations of our maximum likelihood estimator $\boldsymbol{\beta}^{*}$. 
\item[-] Suppose $h(\boldsymbol{\beta}^{*})$ is a scalar.    
\item[-] Call $h(\boldsymbol{\beta}^{*})_{\alpha}$ the value such that $\alpha$ of values less than or equal to it.  (For example $h(\boldsymbol{\beta}^{*})_{0.5}$ is the \alert{median} (value such that 50\% lower)
\item[-] A $1 - \alpha$ confidence interval (under some additional assumptions) is 
\end{itemize}

\begin{eqnarray}
\text{CI}_{1-\alpha} & = & \left(h(\boldsymbol{\beta}^{*})_{\alpha/2}, h(\boldsymbol{\beta}^{*})_{1- \alpha/2}\right) \nonumber
\end{eqnarray}


\end{frame}


\begin{frame}
Recall:
\begin{semiverbatim}
store\_expected<- c()

for(z in 1:1000)\{

\hspace{0.2in}	subset<- sample(1:nrow(mips), nrow(mips), replace=T)

\hspace{0.2in}	use\_mips<- mips[subset,]

\hspace{0.2in}	temp\_reg<- glm(mip$\sim$treat + gop, 

\hspace{0.4in}		family = binomial(link='probit'), 

\hspace{0.4in}		data = as.data.frame(use\_mips))

\hspace{0.2in}	store\_expected[z]<- pnorm(temp\_reg\$coef\%*\%X\_synth)

	\}

>conf\_int\_95<- quantile(store\_expected, c(0.025, 0.975))

> conf\_int\_95

     2.5\%       97.5\%

0.2953548 0.3546646 

\end{semiverbatim}

\end{frame}


\begin{frame}
\frametitle{Bootstrapping is not a Panacea}

\pause 
\begin{itemize}
\invisible<1>{\item[1)] If you have dependence in your data, you need to sample to preserve it (time series, nested data) (but you can do a block bootstrap)}\pause 
\invisible<1-3>{\item[2)] Bootstrap doesn't always work: (but you can fix it, usually)
\begin{itemize}
\item[-] Matching
\item[-] LASSO
\item[-] Maximum (e.g. MLE for uniform)
\end{itemize}}\pause 
\invisible<1-4>{\item[-] Final lecture: talk more about non-vanilla bootstrap } 
\end{itemize}


\end{frame}



\begin{frame}
\frametitle{Inference Three Ways}

\begin{itemize}
\item[1)] Bootstrap $\leadsto$ (no asymptotics, simulation)
\alert{\item[2)] Delta Method $\leadsto$ (asymptotic normality, analytic)}
\item[3)] Simulation from Multivariate Normal $\leadsto$ (asymptotic normality, simulation)
\end{itemize}


\end{frame}



\begin{frame}
\frametitle{Delta Method (Procedure, Then Justification)}

Suppose we have maximum likelihood estimator $\boldsymbol{\beta}^{*}$. \pause  \\
\invisible<1>{We are interested in $h(\boldsymbol{\beta}^*)$.  (Assuming a large sample):} \pause 

\begin{eqnarray}
\invisible<1-2>{h(\boldsymbol{\beta}^{*}) & \longrightarrow^{D} & \text{Multivariate Normal} \left(h(\boldsymbol{\beta}), \nabla h(\boldsymbol{\beta}^{*})^{'}I_{N}(\boldsymbol{\beta}^{*})^{-1} \nabla h(\boldsymbol{\beta}^{*}) \right) \nonumber } \pause 
\end{eqnarray} 

\invisible<1-3>{Inference for $h(\boldsymbol{\beta}^{*}) $} \pause 
\begin{itemize}
\invisible<1-4>{\item[1)] Apply function $h$ to maximum likelihood estimates:} \pause 
\begin{eqnarray}
\invisible<1-5>{h(\boldsymbol{\beta}^{*}) & = & \Phi(\boldsymbol{X}_{i}^{'}\boldsymbol{\beta}^{*}) \nonumber } \pause 
\end{eqnarray} 
\invisible<1-6>{\item[2)] To calculate variance-covariance matrix:} \pause 
\begin{itemize}
\invisible<1-7>{\item[a)] Calculate $\nabla h(\boldsymbol{\beta}^{*}) $ at maximum likelihood estimates:} \pause 
\begin{eqnarray}
\invisible<1-8>{\nabla h(\boldsymbol{\beta}^{*})& = & \left(\frac{\partial h(\boldsymbol{\beta})}{\partial \beta_{1} }|_{\boldsymbol{\beta}^{*} }, \frac{\partial h(\boldsymbol{\beta})}{\partial \beta_{2} }|_{\boldsymbol{\beta}^{*} }, \hdots, \frac{\partial h(\boldsymbol{\beta})}{\partial \beta_{K} }|_{\boldsymbol{\beta}^{*} } \right) \nonumber } \pause 
\end{eqnarray}
\invisible<1-9>{\item[b)] Calculate Variance covariance matrix: } \pause 
\begin{eqnarray}
\only<11>{\invisible<1-10>{\nabla h(\boldsymbol{\beta}^{*})^{'}I_{N}(\boldsymbol{\beta}^{*})^{-1} \nabla h(\boldsymbol{\beta}^{*}) && \nonumber }} 
\only<12>{\nabla h(\boldsymbol{\beta}^{*})^{'}(-E[\text{Hessian}]^{-1}) \nabla h(\boldsymbol{\beta}^{*}) && \nonumber}
\end{eqnarray}

\end{itemize}


\end{itemize}


\end{frame}

\begin{frame}
\frametitle{Example:}

\begin{eqnarray}
\frac{\partial E[Y| \boldsymbol{X}_{i}, \boldsymbol{\beta}]}{ \partial \boldsymbol{\beta}_{k}} & = & \phi(\boldsymbol{X}_{i} \boldsymbol{\beta})X_{k} \nonumber 
\end{eqnarray}

\begin{semiverbatim}

prob\_grad<- function(X, coef)\{

\hspace{0.3in}	base<- dnorm(coef\%*\%X)

\hspace{0.3in}	out<- base*X

\hspace{0.3in}	return(out)
	
\hspace{0.2in}\}

> mle\_grad<- prob\_grad(X\_synth, reg1\$coef)


> mle\_grad

[1] 0.359335 0.000000 0.359335

 
> ses<- sqrt(t(mle\_grad)\%*\%vcov(reg1)\%*\%mle\_grad)

 
> ses
           
[1,] 0.01618305


\end{semiverbatim}






\end{frame}


\begin{frame}
\frametitle{Comparison}

\begin{tabular}{ccc}
\hline
	& Bootstrap & Delta Method \\
\hline	
Standard Error &  0.016		& 	0.016	\\
Confidence Interval & [0.295, 0.355]	  & 		\\
\hline
\end{tabular}




\end{frame}





\begin{frame}
\frametitle{Delta Method (Justification)}

Recall that 

\pause 
\invisible<1>{\begin{eqnarray}
\boldsymbol{\beta}^{*} & \longrightarrow^{D} & \text{Multivariate Normal}\left(\boldsymbol{\beta}, I(\boldsymbol{\beta}^{*})^{-1}\right) \nonumber 
\end{eqnarray}}\pause 

\invisible<1-2>{Multivariate Taylor Series: } \pause 

\invisible<1-3>{\begin{eqnarray}
h(\boldsymbol{\beta}^{*} ) & = & h(\boldsymbol{\beta}) + \nabla h(\boldsymbol{\beta})^{'}(\boldsymbol{\beta}^{*} - \boldsymbol{\beta})  + \text{ Remainder } \nonumber 
\end{eqnarray}} \pause 


\end{frame}

\begin{frame}
\frametitle{Delta Method (Justification)}


Take the limit: 

\begin{eqnarray}
\lim_{n \rightarrow \infty} h(\boldsymbol{\beta}^{*} ) & = & \lim_{n \rightarrow \infty} \left(h(\boldsymbol{\beta}) + \nabla h(\boldsymbol{\beta})^{'}(\boldsymbol{\beta}^{*} - \boldsymbol{\beta})  + \text{ Remainder } \right) \nonumber \\
\lim_{n \rightarrow \infty} h(\boldsymbol{\beta}^{*} )& \longrightarrow^{P} & h(\boldsymbol{\beta}) \nonumber \\
\lim_{n \rightarrow \infty} h(\boldsymbol{\beta}^{*} )& \longrightarrow^{D} & h(\boldsymbol{\beta}) + \lim_{n \rightarrow \infty} \left(\nabla 
h(\boldsymbol{\beta})^{'}
\underbrace{(\boldsymbol{\beta}^{*} - \boldsymbol{\beta})}_{\text{MVN}(0, I(\boldsymbol{\beta})^{-1}} \right)  \nonumber 
\end{eqnarray}
\pause 

\invisible<1>{
Then by Slutsky's Theorem:


\begin{eqnarray}
h(\boldsymbol{\beta}^{*} ) & \longrightarrow^{D} & \text{MVN}(h(\boldsymbol{\beta}), \nabla h(\boldsymbol{\beta}^{*})^{'}I_{N}(\boldsymbol{\beta}^{*})^{-1} \nabla h(\boldsymbol{\beta}^{*})) \nonumber 
\end{eqnarray}
}
\end{frame}


\begin{frame}
\frametitle{Delta Method Confidence Intervals}

\begin{itemize}
\item[-] Suppose $h(\boldsymbol{\beta}^{*}) $ is a scalar\\
\item[-] SE($h(\boldsymbol{\beta}^{*} )) = \sqrt{\left(\nabla h(\boldsymbol{\beta}^{*})^{'}I_{N}(\boldsymbol{\beta}^{*})^{-1} \nabla h(\boldsymbol{\beta}^{*})\right) }$
\item[-] Then a $1- \alpha$ confidence interval is:
\end{itemize}

\begin{eqnarray}
\text{CI}_{1-\alpha} & = & \left[h(\boldsymbol{\beta}^{*} )- 1.96 \times \text{SE}(h(\boldsymbol{\beta}^{*} )) , h(\boldsymbol{\beta}^{*} ) + 1.96 \times \text{SE}(h(\boldsymbol{\beta}^{*} ) ) \right] \nonumber 
\end{eqnarray}


\end{frame}


\begin{frame}
\frametitle{Delta Method: Example}


\begin{semiverbatim}

> y.prob<- pnorm(y.tilde)

> X\_synth<- c(1, 0, 1)

> y.tilde<- reg1\$coef\%*\%X\_synth

> y.prob<- pnorm(y.tilde)

> y.prob
          
[1,] 0.3237277


> ses<- sqrt(t(mle\_grad)\%*\%vcov(reg1)\%*\%mle\_grad)

> ses
         
[1,] 0.01618305

> delta\_95<- c(y.prob - 1.96*ses, y.prob + 1.96*ses)

> delta\_95

[1] 0.2920089 0.3554464

\end{semiverbatim}


\end{frame}


\begin{frame}
\frametitle{Comparison}

\begin{tabular}{ccc}
\hline
	& Bootstrap & Delta Method \\
\hline	
Standard Error &  0.016		& 	0.016	\\
Confidence Interval & [0.295, 0.355]	 & [0.292, 0.355] \\
\hline
\end{tabular}



\end{frame}



\begin{frame}
\frametitle{Inference Three Ways}

\begin{itemize}
\item[1)] Bootstrap $\leadsto$ (no asymptotics, simulation)
\item[2)] Delta Method $\leadsto$ (asymptotic normality, analytic)
\alert{\item[3)] Simulation from Multivariate Normal $\leadsto$ (asymptotic normality, simulation)} 
\end{itemize}


\end{frame}


\begin{frame}
\frametitle{Simultation from Multivariate Normal (Procedure)}

$\boldsymbol{\beta}^{*} \longrightarrow^{D} \text{MVN}(\boldsymbol{\beta}, I(\boldsymbol{\beta})^{-1}) $

\pause 



\invisible<1>{Inference about $h(\boldsymbol{\beta}^{*})$: } \pause \\
\invisible<1-2>{For each simulation $m$ of $M$ simulations:} \pause 
\begin{eqnarray}
\invisible<1-3>{\boldsymbol{\beta}^{m} & \sim & \text{MVN}\left(\boldsymbol{\beta}^{*}, I(\boldsymbol{\beta})^{-1} \right) \nonumber\\} \pause 
\invisible<1-4>{h(\boldsymbol{\beta})^{m} & = & h(\boldsymbol{\beta}^{m}) \nonumber } \pause 
\end{eqnarray}

\invisible<1-5>{Use collection of simulated values $\left(h(\boldsymbol{\beta})^{1}, h(\boldsymbol{\beta})^{2}, \hdots, h(\boldsymbol{\beta})^{M}\right)$ to approximate distribution of $h(\boldsymbol{\beta}^{*})$.  } \pause 

\begin{eqnarray}
\only<1-7>{\invisible<1-6>{E[h(\boldsymbol{\beta}^{*})] & = & \sum_{m=1}^{M} \frac{h(\boldsymbol{\beta})^{m} }{M} \nonumber } } 
\only<8>{\text{var}[h(\boldsymbol{\beta}^{*})] & = & \sum_{m=1}^{M} \frac{\left(h(\boldsymbol{\beta})^{m} -  \sum_{m=1}^{M} \frac{h(\boldsymbol{\beta})^{m} }{M}  \right)^{2} }{M} \nonumber } 
\only<9>{f[h(\boldsymbol{\beta}^{*})] & = & \sum_{m=1}^{M} \frac{ f(h(\boldsymbol{\beta})^{m}) }{ M } \nonumber }
\end{eqnarray}

\end{frame}

\begin{frame}
\frametitle{Simulation: Example}


\begin{semiverbatim}
libary(MASS)

draw\_coef<- mvrnorm(1000, mu = reg1\$coef, Sigma = vcov(reg1))

> dim(draw\_coef)

[1] 1000    3


dist\_exp<- pnorm(draw\_coef\%*\%X\_synth)

> length(dist\_exp)

[1] 1000


sd(dist\_exp)

[1] 0.01572543

\end{semiverbatim}


\end{frame}


\begin{frame}
\frametitle{Comparison}

\begin{tabular}{cccc}
\hline
	& Bootstrap & Delta Method & MVN Simulation \\
\hline	
Standard Error      &  0.016			 & 	0.016	      &  0.016\\
Confidence Interval & [0.295, 0.355]	 & [0.292, 0.355] &   \\
\hline
\end{tabular}



\end{frame}



\begin{frame}
\frametitle{Simultation from Multivariate Normal (Justification)}


Simulation is a (stronger assumption) double approximation

\begin{eqnarray}
E_{f}[h(\boldsymbol{\beta})^{*} ] & = & \int h(\boldsymbol{\beta})^{*} f(\boldsymbol{\beta}^{*})d\boldsymbol{\beta}^{*} \nonumber \\
& \approx & \int h(\boldsymbol{\beta}^{*}) \underbrace{\tilde{f}(\boldsymbol{\beta}^{*})}_{\text{MVN}}d\boldsymbol{\beta}^{*} \approx \sum_{m=1}^{M} \frac{h(\boldsymbol{\beta})^{m} }{M} \nonumber 
\end{eqnarray}


\end{frame}


\begin{frame}
\frametitle{Percentile Simulation Confidence Intervals}

\begin{itemize}
\item[-] Suppose we have $M$ draws 
\item[-] Suppose $h(\boldsymbol{\beta}^{*})$ is a scalar
\item[-] Then a $1-\alpha$ confidence interval is:
\begin{eqnarray}
\text{CI}_{1-\alpha} & = & \left[h(\boldsymbol{\beta}^{*})_{\alpha/2}, h(\boldsymbol{\beta}^{*})_{1 - \alpha/2}\right] \nonumber 
\end{eqnarray}
\item[-] In practice: order the values, select quantile values at $\alpha/2$ and $1- \alpha/2$.  
\end{itemize}


\end{frame}


\begin{frame}
\frametitle{Simulation Confidence Intervals}



\begin{semiverbatim}

sim\_95<- quantile(dist\_exp, c(0.025, 0.975))

> sim\_95

     2.5\%     97.5\% 

0.2953568 0.3559635 

\end{semiverbatim}



\end{frame}



\begin{frame}
\frametitle{Comparison}

\begin{tabular}{cccc}
\hline
	& Bootstrap & Delta Method & MVN Simulation \\
\hline	
Standard Error      &  0.016			 & 	0.016	      &  0.016\\
Confidence Interval & [0.295, 0.355]	 & [0.292, 0.355] & [0.295, 0.356]   \\
\hline
\end{tabular}



\end{frame}




\begin{frame}
\frametitle{So Which Should I Use?}

Lots of data
\begin{itemize}
\item[-] Then all three methods will be close
\item[-] Delta Method + MVN Simulation $\leadsto$ faster (but what else are you doing?)
\end{itemize}

Not lots of data

\begin{itemize}
\item[-] Then only bootstrap will perform well 
\item[-] And will be comparable in speed to Delta + MVN
\end{itemize}




\end{frame}


\begin{frame}
\frametitle{Wendesday}


Ordered Probit


\end{frame}



\end{document} 