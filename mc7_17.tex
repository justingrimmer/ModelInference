\documentclass{beamer}

%\usepackage[table]{xcolor}
\mode<presentation> {
  \usetheme{Boadilla}
%  \usetheme{Pittsburgh}
%\usefonttheme[2]{sans}
\renewcommand{\familydefault}{cmss}
%\usepackage{lmodern}
%\usepackage[T1]{fontenc}
%\usepackage{palatino}
%\usepackage{cmbright}
  \setbeamercovered{transparent}
\useinnertheme{rectangles}
}
%\usepackage{normalem}{ulem}
%\usepackage{colortbl, textcomp}
\setbeamercolor{normal text}{fg=black}
\setbeamercolor{structure}{fg= black}
\definecolor{trial}{cmyk}{1,0,0, 0}
\definecolor{trial2}{cmyk}{0.00,0,1, 0}
\definecolor{darkgreen}{rgb}{0,.4, 0.1}
\newtheorem{prop}{Proposition}
\newtheorem{defn}{Definition}
\newtheorem{cor}{Corollary}
\newtheorem{lem}{Lemma}

\newcommand\indep{\protect\mathpalette{\protect\independenT}{\perp}}
\def\independenT#1#2{\mathrel{\rlap{$#1#2$}\mkern2mu{#1#2}}}

\usefonttheme{professionalfonts} % using non standard fonts for beamer
\usefonttheme{serif}

\beamertemplatesolidbackgroundcolor{white}  \setbeamercolor{alerted
text}{fg=red}
\usepackage{multirow}

\setbeamertemplate{caption}[numbered]\newcounter{mylastframe}

%\begin{beamerboxesrounded}[upper=uppercol,lower=lowercol,shadow=true]{Block}
%$A = B$.
%\end{beamerboxesrounded}}
\renewcommand{\familydefault}{cmss}
\usepackage[all]{xy}
\usepackage{tikz}
\usetikzlibrary{arrows}
%\usepackage{palatino}
%\usepackage{eulervm}
\usecolortheme{lily}
%\usepackage[latin1]{inputenc}

\title[Methodology III] % (optional, nur bei langen Titeln nötig)
{Political Methodology III: Model Based Inference}

\author{Justin Grimmer}
\institute[Stanford University]{Associate Professor\\Department of Political Science \\  Stanford University}
\vspace{0.3in}

\date{April 24th, 2017}

\begin{document}
\begin{frame}
\titlepage
\end{frame}


\begin{frame}
\frametitle{Model Based Inference}

\begin{itemize}
\item[1)] Likelihood inference
\item[2)] Logit/Probit
\item[3)] Ordered Probit
\item[4)] Choice Models:
\begin{itemize}
\item[-] Multinomial Logit
\begin{itemize}
\item[a)] DGP
\item[b)] IIA
\item[c)] Optimization
\item[d)] Quantities of Interest
\item[e)] Interpretation
\end{itemize}
\item[-] Multinomial Probit
\begin{itemize}
\item[a)] DGP
\item[b)] No IIA, But No Likelihood
\item[c)] Quantities of Interest
\item[d)] Interpretation
\end{itemize}
\end{itemize}
\end{itemize}

\end{frame}



\begin{frame}
\frametitle{Modeling Categorical Outcomes}
Categorical outcome (a.k.a. discrete choice) variable:
    $$ Y_i \ \in \ \{1, 2,..., J\}$$
\pause
Example: Multiparty elections
\begin{itemize}
\item Voters choose from more than two parties/candidates
\item $1=$ Bush; $2=$ Clinton; $3=$ Perot
\item $1=$ Trump; $2=$ Cruz; $3=$ Kasich; $4=$ Rubio; $5=$ Bush ...
\end{itemize}
\pause
\medskip
Possible research questions:
  \begin{itemize}
  \item How are candidate characteristics associated with
    probability of voting? (e.g. past experience, campaign spending)
  \item What kind of voters tend to choose Perot instead of Bush
    or Clinton? (e.g. age, race, ideology)
  \item How would vote shares have changed if Perot had not run?
  \end{itemize}
\medskip
\pause
Goal: Model probability of choosing $Y_i = j$ as function of predictors

\end{frame}

%%%%%%%%% frame %%%%%%%%%
\begin{frame}
\frametitle{Example: 1992 U.S. Presidential Election}
Alvarez and Nagler (1995):
\begin{itemize}
\item $Y_i$: Vote choice in the 1992 U.S. presidential election \\
($1 =$ Clinton, $2 = $ Bush, $3 = $ Perot)
\end{itemize}
\vspace{-\baselineskip}
\begin{center}
\includegraphics[width=.7\textwidth]{1992pres-hist}
\end{center}
\pause
\vspace{-2\baselineskip}
\begin{itemize}
\item Two types of predictors:
	\smallskip
	\begin{itemize}
	\addtolength{\itemsep}{.25\baselineskip}
	\item Voter-specific ($V_i$): age, gender, education, party, opinions, etc.
	\item Candidate-varying ($X_{ij}$): ideological distance between voter $i$ and candidate $j$
	\end{itemize}
\end{itemize}
\end{frame}





\begin{frame}

\begin{defn}
Suppose we observe a trial, $i$, which might result in $J$ outcomes.  \\
And that P$(\text{outcome } = i) = \pi_{i}$ \\
$Y_{i} = j$  if outcome $j$ occurred and 0 otherwise.

Then $Y_{i}$ follows a \alert{multinomial} distribution, with \\
\begin{eqnarray}
p(y_{i} ) & = & \pi_{1}^{I(y_{i}= 1) } \pi_{2}^{I(y_{i} = 2) } \hdots \pi_{J}^{I(y_{i} = J)} \nonumber \\
p(y_{i} ) & = & \prod_{j=1}^{J} \pi_{j}^{I(y_{j} = j ) } \nonumber
\end{eqnarray}
Equivalently, we'll write
\begin{eqnarray}
Y_{i}  & \sim & \text{Multinomial}(1, \boldsymbol{\pi}) \nonumber \\
Y_{i} & \sim & \text{Categorical}(\boldsymbol{\pi}) \nonumber
\end{eqnarray}
\end{defn}

\end{frame}


\begin{frame}
\frametitle{Multinomial Distribution}

\pause
\begin{eqnarray}
\invisible<1>{\text{E}[I(Y_{i} = j) ] & = & \pi_{j} \nonumber } \pause \\
\invisible<1-2>{\text{E}[I(Y_{i} = J) ] & = & \pi_{J} = 1 - \pi_{1} - \pi_{2} - \hdots - \pi_{J-1} \nonumber} \pause  \\
\invisible<1-3>{\text{Var}[I(Y_{i} = j) ] & = & \pi_{j}(1 - \pi_{j} ) \nonumber } \pause \\
\invisible<1-4>{\text{Cov}(I(Y_{i} = j), I(Y_{i}= k) ]  & = & - \pi_{j} \pi_{k} \nonumber}
\end{eqnarray}

\end{frame}



\begin{frame}
\frametitle{Vote Choice: Parameters of Multinomial Distribution}

Suppose we make $N$ independent draws:
\begin{eqnarray}
Y_{i} & \sim_{\text{iid}} & \text{Multinomial}(1, \boldsymbol{\pi})  \nonumber
\end{eqnarray}

\pause

\invisible<1>{Then:}
\begin{eqnarray}
\invisible<1>{L(\boldsymbol{\pi}| \boldsymbol{Y} ) & = & f(\boldsymbol{Y}|\boldsymbol{\pi}) \nonumber \\} \pause
					\invisible<1-2>{				 & = & \prod_{i=1}^{N} f(Y_{i}|\boldsymbol{\pi}) \nonumber \\} \pause
									 \invisible<1-3>{& = & \prod_{i=1}^{N}\prod_{j=1}^{J} \pi_{j}^{I(Y_{i} = j) } \nonumber }
\end{eqnarray}

\end{frame}

\begin{frame}
\frametitle{Vote Choice}

Obtaining maximum-likelihood estimates:

\begin{eqnarray}
L(\boldsymbol{\pi}| \boldsymbol{Y} ) & = & \prod_{i=1}^{N}\prod_{j=1}^{J} \pi_{j}^{I(Y_{i}= j)} \nonumber \pause  \\
\invisible<1>{\ell (\boldsymbol{\pi}| \boldsymbol{Y} ) & = & \sum_{i=1}^{N} \sum_{j=1}^{J} I(Y_{i}=j) \log \pi_{j} \nonumber } \pause
\end{eqnarray}

\invisible<1-2>{Include constraint that $\sum_{j=1}^{J} \pi_{j} = 1$ } \pause

\begin{eqnarray}
\invisible<1-3>{\ell (\boldsymbol{\pi}| \boldsymbol{Y} ) & = & \sum_{i=1}^{N} \sum_{j=1}^{J} I(Y_{i}=j) \log \pi_{j} + \lambda(\sum_{j=1}^{J} \pi_{j} - 1)  \nonumber }
\end{eqnarray}



\end{frame}


\begin{frame}
\frametitle{Vote Choice}


\begin{eqnarray}
\ell(\boldsymbol{\pi}| \boldsymbol{Y} ) & = & \sum_{i=1}^{N} \sum_{j=1}^{J} I(Y_{i}=j) \log \pi_{j} + \lambda(\sum_{j=1}^{J} \pi_{j} - 1)  \nonumber  \\
\frac{\partial \ell(\boldsymbol{\pi}| \boldsymbol{Y} )}{\partial \pi_{1}} & = & \frac{ \sum_{i=1}^{N} I(Y_{i}= 1) }{\pi_{1}} + \lambda \nonumber \\
\frac{\partial \ell(\boldsymbol{\pi}| \boldsymbol{Y} )}{\partial \pi_{2}} & = & \frac{ \sum_{i=1}^{N} I(Y_{i}= 2)}{\pi_{2}} + \lambda \nonumber \\
\frac{\partial \ell(\boldsymbol{\pi}| \boldsymbol{Y} )}{\partial \pi_{3}} & = & \frac{ \sum_{i=1}^{N} I(Y_{i}= 3)}{\pi_{3}} + \lambda \nonumber \\
\vdots & \vdots & \vdots  \nonumber \\
\frac{\partial \ell(\boldsymbol{\pi}| \boldsymbol{Y} )}{\partial \pi_{J}} & = & \frac{ \sum_{i=1}^{N} I(Y_{i}= J)}{\pi_{J}} + \lambda \nonumber \\
\frac{\partial \ell(\boldsymbol{\pi}| \boldsymbol{Y} )}{\partial \lambda }& = & \sum_{j=1}^{J}\pi_{j} - 1 \nonumber
\end{eqnarray}
\end{frame}

\begin{frame}
\frametitle{Vote Choice}

\huge
Set equal to zero \\
Solve for $\boldsymbol{\pi}^{*}  = (\pi_{1}^{*}, \pi_{2}^{*}, \hdots, \pi_{J}^{*} ) $


\end{frame}



\begin{frame}
\frametitle{Vote Choice}

Note that:
$\sum_{i=1}^{N} \sum_{j=1}^{J} I(y_{i}=j) = N $
\pause
\begin{eqnarray}
\invisible<1>{\pi_{1}^{*} & = & \frac{ \sum_{i=1}^{N} I(Y_{i}=1) }{N} \nonumber \\} \pause
\invisible<1-2>{\pi_{2}^{*} & = & \frac{\sum_{i=1}^{N} I(Y_{i}=2)}{N } \nonumber \\
\pi_{3}^{*} & = & \frac{\sum_{i=1}^{N} I(Y_{i}=3)}{N } \nonumber \\
\vdots & \vdots & \vdots \nonumber \\
\pi_{J}^{*} & = & \frac{\sum_{i=1}^{N} I(Y_{i}=J)}{N } \nonumber }\pause
\end{eqnarray}


\invisible<1-3>{Maximum likelihood estimates$\leadsto$ Average proportion of time candidate is chosen}


\end{frame}


\begin{frame}

\huge

Three Candidate Choice



\end{frame}




\begin{frame}


\begin{center}
\only<1>{\scalebox{0.55}{\includegraphics{Simplex3d.pdf}}}
\end{center}


\end{frame}


\begin{frame}

\begin{tikzpicture}
\node (empty) at (-8, 5) [] {} ;

\node (figure) at (-7, 7) [] {\scalebox{0.55}{\includegraphics{TernaryPlot1.pdf}}};

\invisible<1, 3->{\node (corner) at (-9.75, 4.3) [] {};
\node (corner2) at (-11, 5) [] {};

\draw[->, line width = 2pt]  (corner2) to [out = 180, in = 90] (corner) ;
\node(corner1) at (-11, 6.25) [] {\scalebox{0.25}{\includegraphics{Corner2.pdf}}};
}

\invisible<1-2, 4->{\node (corner3) at (-6.65, 9.7) [] {};
\node (corner4) at (-4.75, 10.2) [] {};

\draw[->, line width = 2pt]  (corner4) to [out = 180, in = 90] (corner3) ;
\node(corner_pic) at (-4, 10.25) [] {\scalebox{0.25}{\includegraphics{Corner1.pdf}}};
}

\invisible<1-3, 5->{\node (corner5) at (-3.7, 4.3) [] {};
\node (corner6) at (-2.5, 5) [] {};

\draw[->, line width = 2pt]  (corner6) to [out = 180, in = 90] (corner5) ;
\node(corner_pic2) at (-2.5, 6.25) [] {\scalebox{0.25}{\includegraphics{Corner3.pdf}}};
}


\invisible<1-4>{\node (corner7) at (-6.7, 6.3) [] {};
\node (corner8) at (-10.5, 10) [] {};

\draw[->, line width = 2pt]  (corner8) to [out = 270, in = 90] (corner7) ;
\node(corner_center) at (-10.5, 10.25) [] {\scalebox{0.25}{\includegraphics{Corner4.pdf}}};
}

\end{tikzpicture}
\pause \pause \pause \pause

\end{frame}


\begin{frame}
\huge
Modeling Choices with Covariates

\end{frame}



\begin{frame}
\frametitle{Multinomial Logit Model}
\begin{itemize}
   \item Generalize the logit model to more than two choices
\bigskip
\pause
   \item The \alert{multinomial logit model (MNL)}:
    $$ \pi_{ij} \ = \ \Pr(Y_i = j\mid V_i) \ = \
    \frac{\exp(V_{i}^{'}\delta_j)}{\sum_{k=1}^J \exp(V_{i}^{'}\delta_k)}, $$
   where $V_i$ $=$ \alert{individual-specific characteristics} of unit $i$ (and an intercept)

\bigskip
\pause
   \item Note that $\sum_{j=1}^J \pi_{ij} = 1$
\medskip
\pause
   \item Need to set the \alert{base category} for identifiability:
     $\delta_1 = 0$
\medskip
\pause
   \item $\delta_j$ represents how characteristics of voter $i$ is
     associated with probability of voting for candidate $j$
\end{itemize}
\end{frame}


\begin{frame}
\frametitle{Multinomial Logit Model: Confirming the Generalization}


Suppose $J = 2$ \pause

\begin{eqnarray}
 \invisible<1>{\pi_{i1}  & =  &  \Pr(Y_i = 1\mid V_i) \nonumber \\
    & = & \frac{\exp(V_{i}^{'} \delta_1)}{\exp(V_{i}^{'} \delta_1) + \exp(V_{i}^{'} \delta_2)} \nonumber} \pause
\end{eqnarray}
\invisible<1-2>{Set $\delta_{2} = 0$ } \pause
\begin{eqnarray}
  \invisible<1-3>{\pi_{i1} & = & \frac{\exp(V_{i}^{'} \delta_1)}{\exp(V_{i}^{'} \delta_1) + 1} \nonumber \\
  & = & \frac{1}{1 + \exp(-V_{i}^{'} \delta_{1} ) } \nonumber  }
\end{eqnarray}

\end{frame}

\begin{frame}
\frametitle{Conditional Logit Model}
\begin{itemize}
 \item We can also incorporate \alert{alternative-varying predictors}
    $X_{ij}$
\pause
  \item The \alert{conditional logit (CL)} model:
    $$ \pi_{ij} \ = \ \Pr(Y_i = j\mid X_{ij}) \ = \
    \frac{\exp(X_{ij}^{'}\beta)}{\sum_{k=1}^J
      \exp(X_{ik}^{'}\beta)} $$
\pause
  \item $\beta$ represents how characteristics of candidate $j$ for voter
    $i$ are associated with voting probabilities
\medskip
\pause
  \item $X_{ij}$ does not have to vary across voters (e.g. whether candidate $j$ is incumbent)
  \item In that case we suppress the subscript to $X_j$
\end{itemize}
\end{frame}


\begin{frame}
\frametitle{MNL as a Special Case of CL}
\begin{itemize}
 \item MNL can be written using CL: Create a set of artificial alternative-varying regressors for each $V_i$:
{\small
   $$ X_{i1} \ = \ \left(\begin{array}{c} V_i \\ 0 \\ \vdots \\ 0 \end{array} \right), \quad
   X_{i2} \ = \ \left(\begin{array}{c} 0 \\ V_i \\ \vdots \\ 0 \end{array} \right),\quad \cdots, \quad
   X_{iJ} \ = \ \left(\begin{array}{c} 0 \\ 0 \\ \vdots \\ V_i \end{array} \right)$$
}
  \item Set the element of $\beta$ for $X_{ij}$ to $\delta_j$ and you get the MNL model
  \item $\delta_1$ must be set to zero for identifiability
\medskip
\pause
  \item Thus we can write both models (and their mixture) simply as CL:
    $$ \pi_{ij} \ = \ \Pr(Y_i = j\mid X) \ = \
    \frac{\exp(X_{ij}^{'}\beta)}{\sum_{k=1}^J
      \exp(X_{ik}^{'}\beta)} $$
  \item We use the names CL and MNL interchangeably from here on
\end{itemize}
\end{frame}

%%%%%%%%% frame %%%%%%%%%
\begin{frame}[fragile]
\frametitle{Predictor Types and Data Formats}
\small

Discrete choice data usually come in one of the two formats:
\begin{enumerate}
\item Wide format: $N$ rows, $\text{\#}V + J\cdot\text{\#}X$ predictors
\begin{verbatim}
 choice women educ idist.Clinton idist.Bush idist.Perot
   Bush     1    3        4.0804     0.1024      0.2601
   Bush     1    4        4.0804     0.1024      0.2601
Clinton     1    2        1.0404     1.7424      0.2401
   Bush     0    6        0.0004     5.3824      2.2201
Clinton     1    3        0.9604    11.0220      6.2001 ...
\end{verbatim}

\item Long format: $NJ$ rows, $\text{\#}V + \text{\#}X$ predictors
\begin{verbatim}
chid     alt choice women educ  idist
   1    Bush   TRUE     1    3 0.1024
   1 Clinton  FALSE     1    3 4.0804
   1   Perot  FALSE     1    3 0.2601
   2    Bush   TRUE     1    4 0.1024
   2 Clinton  FALSE     1    4 4.0804
   2   Perot  FALSE     1    4 0.2601 ...
\end{verbatim}
\end{enumerate}

\end{frame}


\begin{frame}
\frametitle{Predictor Types and Data Formats}

\begin{itemize}
 \item Use {\tt reshape} to change between wide and long
 \item Some estimation functions (e.g. {\tt mlogit}) can take both formats
\end{itemize}


\end{frame}


%%%%%%%%% frame %%%%%%%%%
\begin{frame}
\frametitle{Latent Variable Interpretation}
\begin{itemize}
  \item Recall the \alert{random utility model}:
    $$ Y_{ij}^\ast \ = \ X_{ij}^{'} \beta + \epsilon_{ij}, $$
    where $ \left\{\begin{array}{ccl}
      Y_{ij}^\ast & = & \text{latent utility from choosing $j$ for $i$} \\
      \epsilon_{ij} & = & \text{stochastic component of the utility} \end{array} \right.$
\medskip
\pause
  \item Assume that voter chooses the most preferred candidate, i.e.,
    $$ Y_i = j \quad \text{if} \quad
    Y_{ij}^\ast \geq Y_{ij^\prime}^\ast \quad \text{for any} \quad j^\prime \in \{1,...,J\}$$
\vspace{-.5\baselineskip}
\pause
  \item Assuming $\epsilon_{ij} \sim_{\text{iid}}$ \alert{type I extreme value distribution},
    this setup implies MNL (McFadden 1974)
\medskip
\pause
{\small
  \item Proof for $J=2$:
\begin{eqnarray*}
  \Pr(Y_i = 1\mid X) & = & \Pr(Y_{i1}^\ast \geq Y_{i2}^\ast \mid X)\\
  & = & \Pr\left(\epsilon_{i2} - \epsilon_{i1} \leq
  (X_{i1} - X_{i2})^{'}\beta\right) \\
  & = & \frac{\exp\left((X_{i1}-X_{i2})^{'}\beta\right)}
  {1 + \exp\left((X_{i1} - X_{i2})^{'}\beta\right)}
  \ = \ \frac{\exp(X_{i1}^{'}\beta)}{\exp(X_{i1}^{'}\beta) + \exp(X_{i2}^{'}\beta)}
\end{eqnarray*}
}
\end{itemize}

\end{frame}


\begin{frame}
\frametitle{Independence of Irrelevant Alternatives}

\begin{eqnarray}
\frac{\pi_{i1} }{\pi_{i2}} \pause \invisible<1>{& = & \frac{\frac{\exp(X_{i1}^{'}\beta )}{\alert{\sum_{k=1}^{J} \exp(X_{ik} \beta ) }} }{ \frac{\exp(X_{i2}^{'}\beta )}{\alert{\sum_{k=1}^{J} \exp(X_{ik} \beta )}} }}\pause \invisible<1-2>{ =\frac{ \exp(X_{i1}^{'}\beta ) }{ \exp(X_{i2}^{'}\beta ) } \nonumber }\pause
\end{eqnarray}


\invisible<1-3>{\alert{Relative Risk}: does not depend on other choices \\} \pause
\invisible<1-4>{\alert{Multinomial Choice}: series of pairwise comparisons}


\end{frame}




\begin{frame}


\scalebox{0.45}{\includegraphics{VotersGuide.jpg}}




\end{frame}


\begin{frame}
\frametitle{Independence of Irrelevant Alternatives (Red Bus, Blue Bus)}

Trump, Cruz, and Sanders remaining candidates \pause \invisible<1>{(God, help us)} \pause
\begin{itemize}
\invisible<1-2>{\item[-] 50\% support Sanders
\item[-] 25\% support Trump
\item[-] 25\% support Cruz } \pause
\end{itemize}

\invisible<1-3>{Cruz drops out, what happens to voters?} \pause
\begin{itemize}
\invisible<1-4>{\item[-] Intuitively: 50\% Sanders, 50\% Trump} \pause
\invisible<1-5>{\item[-] Under model, Expected latent utilities are:
\begin{eqnarray}
\text{Sanders} & = & \log 2 \nonumber \\
\text{Trump} & = & 0 \nonumber \\
\text{Cruz} & = & 0 \nonumber \\} \pause
\invisible<1-6>{\text{Pr(Sanders)} & = & \frac{\exp(\log 2) }{ \exp(\log 2) + \exp(0) + \exp(0)} = 2/(2 + 1 + 1) = 0.5 \nonumber }
\end{eqnarray}

\end{itemize}

\end{frame}

\begin{frame}
\frametitle{Independence of Irrelevant Alternatives (Red Bus, Blue Bus)}

 But then: \pause
 \begin{eqnarray}
\invisible<1>{\frac{\text{Pr(Sanders)}}{\text{Pr(Trump)}}  & = & \frac{\exp(\log 2) }{\exp(0)} = 2/1 \nonumber} \pause
 \end{eqnarray}

\invisible<1-2>{Or the model predicts voters choose Sanders $2:1 $ over Trump(!) resulting in (66\%,33\%) split}

\end{frame}

\begin{frame}
\frametitle{IIA and Your Work}

\Large

What to do? \pause
\begin{itemize}
\invisible<1>{\item[1)] Alternative models: Multinomial Probit,  Nested Logit} \pause
\invisible<1-2>{\item[2)] Conditional on included covariates (other information on individual voters would limit the problem)} \pause
\invisible<1-3>{\item[3)] Can overstate size of problem: estimation error (standard errors) often much larger than deviation}
\end{itemize}

\end{frame}



\begin{frame}
\frametitle{Estimation and Inference}
\begin{itemize}
  \item Likelihood for a random sample of size $N$:
  \begin{eqnarray}
    \textstyle L(\boldsymbol{\beta} \mid Y, X) & = &  \prod_{i=1}^N \prod_{j=1}^J
    \pi_{ij}^{I(Y_{i}=j) } \nonumber\\
    &= & \prod_{i=1}^N \prod_{j=1}^J \left(\frac{\exp(X_{ij}^{'} \boldsymbol{\beta})   }{\sum_{k=1}^{J} \exp(X_{ik}^{'} \boldsymbol{\beta})  }\right)^{I(Y_{i} = j)}  \nonumber \\
 \textstyle \log L(\boldsymbol{\beta}\mid Y, X)   & = & \sum_{i=1}^{N} \sum_{j=1}^{J} I(Y_{i} = j) \left(X_{ij}^{'}\boldsymbol{\beta}  - \log\left( \sum_{k=1}^{J} \exp(X_{ik}^{'} \boldsymbol{\beta})   \right) \right)\nonumber
   \end{eqnarray}
\end{itemize}
\end{frame}





\begin{frame}
\frametitle{Estimation and Inference}
\begin{itemize}
  \item Score:
    $$  \textstyle s(\boldsymbol{\beta}| Y_{i}, \boldsymbol{X}_{i}) \ = \  \sum_{j=1}^J I(Y_{i}=j)
    \left(X_{ij} - \bar{X_i} \right), $$
    where $\bar{X_i}$ is the weighted average of $X_i$,
    i.e. $\bar{X_i} = \sum_{j=1}^J \pi_{ij}X_{ij}$
\pause
  \item Solve $\sum_{i=1}^{N} s(\boldsymbol{\beta}| Y_{i}, \boldsymbol{X}_{i}) = 0$ (numerically) to get $\boldsymbol{\beta}_{MLE}^{*}$
  \item It can be shown that the log-likelihood is globally concave \\
    $\Rightarrow$ guaranteed convergence to the true (not local) MLE
\medskip

\pause
  \item Information:
    $$ \textstyle I_{N} (\boldsymbol{\beta}) \ = \ -\text{E}[H(\boldsymbol{\beta})] \ = \
    \sum_{i=1}^N \sum_{j=1}^J \pi_{ij}
    \left(X_{ij} - \bar{X_i} \right)\left(X_{ij} - \bar{X_i} \right)^{'} $$

\end{itemize}

\end{frame}

\begin{frame}
\frametitle{Estimation and Inference}
\begin{eqnarray}
\boldsymbol{\beta}^{*}_{MLE} \longrightarrow^{D}
    \text{Normal}\left(\boldsymbol{\beta}, I(\boldsymbol{\beta})^{-1}\right) \nonumber
    \end{eqnarray}




\end{frame}

%%%%%%%%% frame %%%%%%%%%
\begin{frame}
\frametitle{Interpreting MNL/CL Coefficients}
In MNL/CL, $\beta$ itself is not necessarily informative about the effect
  of $X$
\bigskip
\pause
\begin{enumerate}
\item The coefficients are all with respect to the baseline
  category
\item[] $\longrightarrow$ Testing $\beta_j = 0$ does not generally make
  sense \\
(unless comparison to the baseline is the goal)

\pause
\medskip

\item Changing $X_{ij}$ has impact on $\Pr(Y_i=k\mid X)$, $k \neq j$:
\pause
\smallskip
	\begin{itemize}
	\addtolength{\itemsep}{.25\baselineskip}
	\item For individual-specific characteristics ($V_i$), even sign of $\delta_j$ may not agree with the direction of the change in response probability for $j$

	\pause

	\item For alternative-varying characteristics ($X_{ij}$), sign of $\beta$ does indicate the direction of the effect, but magnitude is hard to interpret

	\end{itemize}
\end{enumerate}

\pause
\bigskip
Compute a quantity that has a clear substantive interpretation!

\end{frame}
%%%%%%%%%%%%%%%%%%%%%%%%%

%%%%%%%%% frame %%%%%%%%%
\begin{frame}
\frametitle{Calculating Quantities of Interest}
\small
\vspace{-.5\baselineskip}
\begin{enumerate}
\item \alert{Choice probability}:
$$ \pi_j(x) \ = \ \Pr(Y_i=j\mid X=x) $$
e.g. How likely is a female college-educated conservative Republican voter to vote for Perot?
\pause
\item \alert{Predicted vote share}:
$$ p_j(x_1) \ \equiv \ \text{E}\left[I\left(\pi_j(X_{i1}=x_1, X_{i2}) \geq \pi_k(X_{i1}=x_1, X_{i2})
 \text{ for all } k\right) \right] $$
where $X_{i1}$ is the predictor(s) of interest and $X_{i2}$ is all other predictors \\
\smallskip
e.g. What would Perot's vote share be if all voters supported abortion?
\pause
\item \alert{Average partial (treatment) effects}:
$$ \tau_{jk} \ = \ \text{E}\left[\pi_j(T_{ik}=1, T_{i\ast}, W_i) - \pi_j(T_{ik}=0, T_{i\ast}, W_i) \right] $$
where $T_{ik}$ is treatment on candidate $k$, $T_{i\ast}$ is treatment on others, $W_i$ is pre-treatment covariates
	\begin{itemize}
	\item ``Direct effect'' if $j=k$; ``indirect effect'' if $j\neq k$
	\item If $T$ is individual-specific, $\tau_j = \text{E}[\pi_j(T_i=1,W_i)-\pi_j(T_i=0,W_i)]$
	\end{itemize}
\end{enumerate}
\pause
\vspace{-.5\baselineskip}
\begin{itemize}
\item Estimate by plugging in sample analogues (e.g. $\pi_j \to \hat\pi_j$, $\text{E} \to \frac{1}{n}\sum$)
\end{itemize}

\end{frame}

%%%%%%%%% frame %%%%%%%%%
\begin{frame}
\frametitle{Example: 1992 U.S. Presidential Election}
\begin{itemize}
\item Model specification (Alvarez and Nagler 1995):
$$ \pi_{ij} \ = \ \frac{\exp(X_{ij}^{'} \beta + V_i^{'} \delta_j)}
 {\sum_{k=1}^J\exp(X_{ik}^{'} \beta + V_i^{'} \delta_k)} $$
where
\begin{eqnarray*}
 X_{ij} & = & \left\{\text{ideological distance}\right\} \\
 V_{i} & = & \left\{1, \text{issue opinions, party, gender, education, age, ...} \right\}
\end{eqnarray*}
\pause
\item Estimated coefficients:
\begin{eqnarray*}
\hat\beta & = & -0.11 \ (0.02) \\
\hat\delta & = & \left[\hat\delta_\text{Bush} \ \hat\delta_\text{Clinton}\right]
 \ = \ \left[ {\scriptsize
            \begin{array}{cc} 0.67 \ (0.94) & -0.41 \ (0.45) \\
                              -0.52 \ (0.11) & -0.02 \ (0.12)  \\
                              0.54 \ (0.23) & 0.30 \ (0.22) \\
                               \vdots & \vdots \end{array} } \right]
           {\scriptsize
            \begin{array}{l} \text{(intercept)} \\
                             \text{(support abortion)} \\
                             \text{(female)} \\
                             \qquad \vdots \end{array} }
\end{eqnarray*}
\end{itemize}
\end{frame}

%%%%%%%%% frame %%%%%%%%%
\begin{frame}
\frametitle{Example: 1992 U.S. Presidential Election}
\begin{itemize}
\item Estimated choice probabilities for a ``typical'' voter from South:
\vspace{-1.5\baselineskip}
\begin{center}
\includegraphics[width=.6\textwidth]{predchoiceprob}
\end{center}
\vspace{-1.5\baselineskip}
\pause
\item Predicted vote shares if everyone opposed abortion:
\vspace{-1.5\baselineskip}
\begin{center}
\includegraphics[width=.6\textwidth]{predvoteshare}
\end{center}
\vspace{-1.5\baselineskip}
\end{itemize}
\end{frame}

%%%%%%%%% frame %%%%%%%%%
\begin{frame}
\frametitle{Revisiting The IIA Assumption}
\begin{itemize}
  \item IIA (Trump, Cruz, and Sanders)
\medskip
\pause
  \item Formally: MNL assumes $\epsilon_{ij}$ is i.i.d.  $\epsilon_{ij} \indep \epsilon_{ik}$ for $j\neq k$
\medskip
\pause
  \item This implies that unobserved factors affecting $Y_{ij}^\ast$
    are unrelated to those affecting $Y_{ik}^\ast$
\medskip
\pause
  \item When is this assumption plausible?
\pause
\bigskip
  \item Example: Multiparty election with parties R, L1 and L2.
\medskip
\pause
  \item Do voters' unobserved ideological preferences affect Pr($Y_i=$L1) independently
    of their effect on Pr($Y_i=$L2)? \pause Probably not.
\end{itemize}
\end{frame}



%%%%%%%%% frame %%%%%%%%%
\begin{frame}
\frametitle{Multinomial Probit Model}
\begin{itemize}
 \item How can we relax the IIA assumption?
\medskip
\pause
  \item Instead of assuming $\epsilon_{ij}$ to be i.i.d. across alternatives $j$,
    we allow $\epsilon_{ij}$ to be correlated across $j$
    within each voter $i$
\medskip
\pause
 \item \alert{Multinomial probit model (MNP)}: \\
   $$ Y_i^\ast \ = \ X_i^{'} \beta + \epsilon_i \quad \text{where}
   \quad \left\{ \begin{array}{l} \boldsymbol{\epsilon}_i \sim_{\text{iid}} \text{MVN}(0, \Sigma_{J}) \\
         Y_i^\ast = [Y_{i1}^\ast \ \cdots \ Y_{iJ}^\ast]^{'} \\
         X_i = \left[ X_{i1} \ \cdots \ X_{iJ}\right]^{'} \end{array}\right. $$

\medskip
\pause
  \item Restrictions on the model for identifiability:
\pause
    \begin{itemize}
      \item The (absolute) \alert{level} of $Y_i^\ast$ shouldn't matter \\ \pause
      $\longrightarrow$ Subtract the 1st equation from
       all the other equations and work with a system of $J-1$
       equations with $\tilde\epsilon_i \sim_{\text{iid}} \text{MVN} (0, \tilde\Sigma_{J-1})$
\pause
      \item The \alert{scale} of $Y_i^\ast$ also shouldn't matter \\ \pause
      $\longrightarrow$ $\tilde\Sigma_{(1,1)} = 1$
    \end{itemize}
\end{itemize}
\end{frame}

%%%%%%%%% frame %%%%%%%%%
\begin{frame}
\frametitle{Limitations of Multinomial Probit}
\begin{itemize}
  \item MNP has no closed-form expression for the \alert{likelihood}:
{\footnotesize
      $$ \hspace{-.2in} \pi_{ij} = \int_{-\infty}^{-\ddot{X}_{1j}^\top \beta}
        \cdots \int_{-\infty}^{-\ddot{X}_{Jj}^\top\beta}
        \phi(\ddot{\epsilon}_{1j},...,\ddot{\epsilon}_{Jj})
        d\ddot{\epsilon}_{1j} \cdots d\ddot{\epsilon}_{Jj}
        \ \text{where} \ \left\{ \begin{array}{ccl}
          \ddot{X}_{kl} & = & X_{ik} - X_{il} \\
         \ddot{\epsilon}_{kl} & = & \epsilon_{ik} - \epsilon_{il} \end{array} \right.
        $$
}
 \item This makes its estimation computationally costly when $J$ large
 \item Must use numerical approximation (quadratures) or simulation
 methods (maximum simulated likelihood or MCMC)
\medskip
\pause
 \item Moreover, \# of parameters in $\Sigma_J$ increases as $J$ gets large,
 but data contain little information about $\Sigma_J$:
 \medskip
 {\small
 \begin{center}
 \begin{tabular}{l|ccccc} \hline
 $J$ & 3 & 4 & 5 & 6 & 7 \\ \hline
 \# of elements in $\Sigma_J$ & 6 & 10 & 15 & 21 & 28 \\
 \# of parameters identified & 2 & 5 & 9 & 14 & 20 \\ \hline
 \end{tabular}
 \end{center}
 }
\bigskip
\pause
\item Consequently, MNP is only feasible when $J$ is small

\medskip
\pause
\item {\tt MNP} in {\tt R}
\item Alternative solutions: Nested logit
\end{itemize}
\end{frame}


\begin{frame}
\frametitle{Even When You Choose Not To Decide, You Still Have Made a Choice}

Lacy and Burden (1999)\\
Multinomial Probit Model:\\
Three Choices: Bush, Perot, and Clinton\\
Four Choices: Bush, Perot, Clinton, and Abstention\\
What happens if Perot doesn't run?

\begin{tabular}{cccc}
\hline
		& Actual & 3-choice & 4-choice \\
\hline
Bush    & 32.0   & 45.7     & 38.4     \\
Clinton & 48.6   & 54.3     & 61.6     \\
Abstention & 20.9 & -       & 23.7    \\
\hline
\end{tabular}


Perot stole from Clinton!

\end{frame}

\begin{frame}

\frametitle{Model Based Inference}

\begin{itemize}
\item[1)] Likelihood inference
\item[2)] Logit/Probit
\item[3)] Ordered Probit
\alert{\item[4)] Choice Models: }
\alert{\item[5)] Count Models (Poisson, Negative Binomial,...)}
\end{itemize}

\end{frame}


\end{document}
