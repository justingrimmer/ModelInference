\documentclass{beamer}

%\usepackage[table]{xcolor}
\mode<presentation> {
  \usetheme{Boadilla}
%  \usetheme{Pittsburgh}
%\usefonttheme[2]{sans}
\renewcommand{\familydefault}{cmss}
%\usepackage{lmodern}
%\usepackage[T1]{fontenc}
%\usepackage{palatino}
%\usepackage{cmbright}
  \setbeamercovered{transparent}
\useinnertheme{rectangles}
}
%\usepackage{normalem}{ulem}
%\usepackage{colortbl, textcomp}
\setbeamercolor{normal text}{fg=black}
\setbeamercolor{structure}{fg= black}
\definecolor{trial}{cmyk}{1,0,0, 0}
\definecolor{trial2}{cmyk}{0.00,0,1, 0}
\definecolor{darkgreen}{rgb}{0,.4, 0.1}
\newtheorem{prop}{Proposition}
\newtheorem{defn}{Definition}
\newtheorem{cor}{Corollary}
\newtheorem{lem}{Lemma}

\newcommand\indep{\protect\mathpalette{\protect\independenT}{\perp}}
\def\independenT#1#2{\mathrel{\rlap{$#1#2$}\mkern2mu{#1#2}}}

\usefonttheme{professionalfonts} % using non standard fonts for beamer
\usefonttheme{serif}

\beamertemplatesolidbackgroundcolor{white}  \setbeamercolor{alerted
text}{fg=red}
\usepackage{multirow}

\setbeamertemplate{caption}[numbered]\newcounter{mylastframe}

%\begin{beamerboxesrounded}[upper=uppercol,lower=lowercol,shadow=true]{Block}
%$A = B$.
%\end{beamerboxesrounded}}
\renewcommand{\familydefault}{cmss}
\usepackage[all]{xy}
\usepackage{tikz}
\usetikzlibrary{arrows}
%\usepackage{palatino}
%\usepackage{eulervm}
\usecolortheme{lily}
%\usepackage[latin1]{inputenc}

\title[Methodology III] % (optional, nur bei langen Titeln nötig)
{Political Methodology III: Model Based Inference}

\author{Justin Grimmer}
\institute[Stanford University]{Associate Professor\\Department of Political Science \\  Stanford University}
\vspace{0.3in}

\date{May 1st, 2017}

\begin{document}
\begin{frame}
\titlepage
\end{frame}


\begin{frame}
\frametitle{Model Based Inference}

\begin{itemize}
\item[1)] Likelihood inference
\item[2)] Logit/Probit
\item[3)] Ordered Probit
\item[4)] Choice Models:
\item[5)] Count Models
\begin{itemize}
\item[-] Negative Binomial Regression
\begin{itemize}
\item[-] DGP
\item[-] Quantities of Interest
\item[-] Interpetation
\end{itemize}
\end{itemize}
\item[-] Clustered Standard Errors
\item[6)] Survival Models
\end{itemize}

\end{frame}


\begin{frame}
\frametitle{Event Count Outcomes}

\begin{itemize}
\item[-] Outcome: number of times an event occurs
\begin{eqnarray}
Y_{i} & \in & \{0, 1, 2, 3, \hdots, \} \nonumber
\end{eqnarray}
\item[-] Examples:
\begin{itemize}
\item[1)] Number of militarized disputes a country is involved in
\item[2)] Number of times a phrase is used
\item[3)] Number of messages into a Congressional office
\item[4)] Number of votes cast for a particular candidate
\end{itemize}
\item[-] Goal:
\begin{itemize}
\item[-] Model the \alert{rate} at which events occur
\item[-] Understand how an intervention (e.g. country becoming a democracy) affects rate
\item[-] Predict number of future events
\end{itemize}
\end{itemize}

\end{frame}




\begin{frame}
  \frametitle{Overdispersion in Poisson Model}

  \begin{itemize}
  \item The Poisson model assumes $\text{E}(Y_i\mid X_i) = \text{Var}(Y_i\mid X_i)$
  \item But for many count data, $\text{E}(Y_i\mid X_i) < \text{V}(Y_i\mid X_i)$
\pause
  \item Potential sources of overdispersion:
    \begin{enumerate}
    \item unobserved heterogeneity
    \item clustering
    \item contagion or diffusion
    \item (classical) measurement error
    \end{enumerate}

\pause
  \item Underdispersion could occur, but rare

\medskip
\pause
  \item One solution to this is to modify the Poisson model by assuming:
    $$ \text{E}(Y_i\mid X_i) = \lambda_i = \exp(X_i^{'}\beta) \quad \text{and} \quad
    \text{Var}(Y_i\mid X_i) = V_i = \alert{\sigma^2}\lambda_i $$
  \item This is called the \alert{overdispersed Poisson regression} model
\pause
  \item When $\sigma^2 > 1$, this corresponds to  the
   negative binomial regression model
  \end{itemize}

\end{frame}

\begin{frame}
\frametitle{Example: Social Network Survey Data}

\begin{center}
  \includegraphics[scale=0.65]{HowManyData.pdf}\\
  \includegraphics[scale=0.65]{HowManyModel.pdf}
\end{center}
\flushright(Zheng, et al., 2006 {\it JASA})


\end{frame}

\begin{frame}
\frametitle{Negative Binomial Distribution}

Suppose $Y_{i} \in \{0, 1, 2, \hdots, \}$.  If $Y_{i}$ has pmf
\begin{eqnarray}
p(y_{i}) & = & \frac{\Gamma\left(\frac{\lambda}{\sigma^2 -1} + y_{i} \right)}{y_{i}! \Gamma\left(\frac{\lambda}{\sigma^2 -1}  \right) } \left(\frac{\sigma^2 - 1}{\sigma^2}\right)^{y_{i}} \left(\sigma^2\right)^{\frac{-\lambda}{\sigma^2 - 1}} \nonumber
\end{eqnarray}
with $\lambda>0$ and $\sigma^2 >1$


Then we will say
\begin{eqnarray}
Y_{i} & \sim & \text{NegBin}(\lambda, \sigma^2)\nonumber \\
E[Y_{i}] &= & \lambda \nonumber \\
\text{Var}(Y_{i}) & = & \lambda \sigma^2 \nonumber
\end{eqnarray}


\end{frame}

\begin{frame}
\frametitle{Negative Binomial Regression}
Suppose:
\begin{eqnarray}
Y_{i} & \sim & \text{Negative Binomial}(\lambda_{i}, \sigma^2)\nonumber \\
\lambda_{i} & = & \exp(\boldsymbol{X}_{i}^{'}\boldsymbol{\beta}) \nonumber
\end{eqnarray}

This implies a likelihood of:
\begin{eqnarray}
L(\boldsymbol{\beta}|\boldsymbol{X}, \boldsymbol{Y}) & = & f(\boldsymbol{Y}| \boldsymbol{X}, \boldsymbol{\beta}) \nonumber \\
& = & \prod_{i=1}^{N} f(Y_{i}| \boldsymbol{X}_{i}, \boldsymbol{\beta}) \nonumber \\
& = & \prod_{i=1}^{N} \frac{\Gamma\left(\frac{\lambda_{i}}{\sigma^2 -1} + y_{i} \right)}{y_{i}! \Gamma\left(\frac{\lambda_{i} }{\sigma^2 -1}  \right) } \left(\frac{\sigma^2 - 1}{\sigma^2}\right)^{y_{i}} \left(\sigma^2\right)^{\frac{-\lambda_{i}}{\sigma^2 - 1}} \nonumber
\end{eqnarray}

Optimize numerically.  Usual theorems about asymptotic distributions apply.
\end{frame}

\begin{frame}
\frametitle{Negative Binomial Regression}

Negative Binomial Regression:
\begin{itemize}
\item[1)] Careful! Variance is sometimes:
\begin{eqnarray}
\text{Var}(Y_{i}| \boldsymbol{X}_{i}) & = & \lambda_{i}(1 + \sigma^2 \lambda_{i})\nonumber
\end{eqnarray}
\item[2)] Run in {\tt R} using
\end{itemize}

\begin{semiverbatim}
library(MASS)

out<- glm.nb(Y$\sim$X)

\end{semiverbatim}

\end{frame}


\begin{frame}

\huge

Clustered Standard Errors

\end{frame}



\begin{frame}
\frametitle{Misspecification and Quasi-MLE}
\begin{itemize}[<+->]
\item When the model is exactly correct, MLE is the best estimator
\item But your model is usually wrong!
\item What if you assumed $f(Y\mid \theta)$ but the true DGP is $g(Y\mid\theta)$?
\pause
\end{itemize}
\bigskip

Point estimate:
\begin{itemize}[<+->]
\item Generally, $\hat\theta$ maximizing $L_f=f$ is \alert{inconsistent}: $\lim_{n\rightarrow\infty}
    \hat\theta \rightarrow^{p} \theta^\ast \neq \theta$
\item Instead, $\theta^\ast$ minimizes the ``divergence'' between $f$ and $g$, defined as:
$$ \text{E}\left[\log g(Y\mid \theta) - \log f(Y\mid \theta)\right] $$
$\longrightarrow$ ``best possible'' assuming $g$, but no guarantee $\theta^\ast$
is {\it substantively} close to $\theta$
\item This $\hat\theta$ is called the \alert{quasi-maximum likelihood estimator} (QMLE)
\item However, for some models and types of misspecification,
$\hat\theta_{QMLE}$ is still consistent for true $\theta$
\end{itemize}
\end{frame}

\begin{frame}
\frametitle{Robust Standard Errors}
Variance estimate:
\begin{itemize}[<+->]
\item For $\hat\theta_{QMLE}$ we can show
    $$\hat\theta \ \stackrel{{\rm approx.}}{\sim} \
        \text{N}\left(\theta^\ast, \ A^{-1}BA^{-1} \right) $$
where \quad $A = -\text{E}[H(\theta^\ast)]$ \quad and \quad $B = \text{E}[s(\theta^\ast)s(\theta^\ast)^\top]$
\bigskip
\item If $f = g$, $\theta^\ast = \theta$ (consistency) and $A = B$ (information equality) \\
    $\longrightarrow$ We get $\hat\theta_{QMLE} \stackrel{{\rm approx.}}{\sim}
    \text{N}(\theta, A^{-1})$
\item Thus, as expected, $\hat\theta_{QMLE} = \hat\theta_{MLE}$
\bigskip
\item If $f\neq g$, usually $A\neq B$
\item So we need to estimate $A$ and $B$ separately and use a
    \alert{sandwich estimator} for variance:
    $$ \widehat{\text{V}(\hat\theta_{QMLE})} \ = \ \hat{A}^{-1}\hat{B}\hat{A}^{-1}$$
    where \quad $\hat{A} = -\sum_{i=1}^n H_i(\hat\theta)$ \quad and \quad
    $\hat{B} = \sum_{i=1}^n s_i(\hat\theta)s_i(\hat\theta)^\top$
\item This gives the \alert{Huber-White robust standard errors}
\end{itemize}
\end{frame}

\begin{frame}
\frametitle{Standard Errors for Cluster Sampled Data}
\begin{itemize}[<+->]
  \item Suppose that data are collected via \alert{cluster sampling},
    i.e., first sampling $M$ clusters and then $N_m$ within each
    cluster $m$
  \item There may be dependence between units within each cluster
\medskip
  \item The correct likelihood would take into account this
    dependence:
$$ \textstyle l(\theta\mid Y) \ = \ \sum_{m=1}^M f_m(Y_{1m}, ..., Y_{N_m m}| \theta) $$
  \item This joint likelihood will be intractable for most models
    (e.g. logit)
  \item Instead, we could look at a quasi-log-likelihood:
$$ \textstyle l^\ast(\theta\mid Y) \ = \ \sum_{m=1}^M \sum_{i=1}^{N_m}
f_i(Y_i\mid \theta) $$
  And ``fix'' the standard errors by allowing within-cluster correlation:
  $$ \hat{B}^\ast = \sum_{m=1}^M \left\{ \left(\sum_{i=1}^{N_m} s_i(\hat\theta)\right)
  \left(\sum_{i=1}^{N_m} s_i(\hat\theta)\right)^{'}\right\} $$
  \item $\hat{A}^{-1}\hat{B}^\ast\hat{A}^{-1}$ gives the
    \alert{cluster robust standard errors}
  \item Asymptotics is w.r.t. $M$ $\Rightarrow$ may be badly behaved
    when $M$ small

\end{itemize}
\end{frame}




\begin{frame}

\huge
Survival analysis

\end{frame}



\frame{

  \frametitle{What is Survival Analysis?}

  \begin{itemize}
  \item Analyze the length of time spent in a given state
  \item $Y_i \in [0, \infty)$: Duration, ``time to an event''
  \item Suppose $Y_{i}$ has density $f(y)$.
\pause
\medskip
  \item Example: Cabinet duration
    \begin{itemize}
      \item Are cabinets more likely to dissolve early or late?
      \item What factors predict the length of time until dissolution?
\pause
\smallskip
      \item King, Alt, Burns \& Laver (1990 AJPS) \dotfill Exponential model
      \item Warwick \& Easton (1992 AJPS) \dotfill Weibull model
      \item Warwick (1992 AJPS) \dotfill Cox PH model
      \item Diermeier \& Stevenson (1999 AJPS) \dotfill Competing risks model
    \end{itemize}
\pause
\medskip
  \item One of the most sophisticated subfields of statistical modeling, developed
    in multiple disciplines
  \item We will only be able to scratch the surface
 \end{itemize}

}

\frame{

  \frametitle{Survival Function}

  \begin{itemize}
  \item \alert{Survival function}: Probability of surviving at least up to time $y$
    $$S(y) \ \equiv \ \Pr(Y_i > y) \ = \ \int_{y}^{\infty} f(y)dy \ = \ 1 - \int_{0}^{y} f(y)dy \ = \  1-F(y)$$
  \item How likely am I to live at least $y$ years?
\pause
\medskip
  \item Properties:
    \begin{itemize}
      \addtolength{\itemsep}{0.25\baselineskip}
    \item $S(0) = 1$ and $S(\infty) = 0$; monotonically decreasing
\pause
\smallskip
    \item Area under $S(y)$ is the average survival time:
      \begin{eqnarray}
      \text{E}(Y_i) & = &  \int_{0}^{\infty} y f(y) dy  \nonumber \\
      & = & y \left(F(y)|^{\infty}_{0}\right)  - \int_{0}^{\infty} F(y) dy \nonumber \\
      & = & \int_{0}^{\infty} (1- F(y)) dy \nonumber \\
      & = & \int_{0}^{\infty} S(y)dy \nonumber
      \end{eqnarray}
\pause
\vspace{-.75\baselineskip}
  \end{itemize}
  \end{itemize}
}

\begin{frame}
\frametitle{Survival Function}

\begin{itemize}
   \item One-to-one relationships with density and probability:
    \begin{eqnarray*}
     & & f(y) \ = \ -\frac{d}{dy}S(y) \quad \text{and} \quad S(y) \ = \ \int_y^\infty f(t) dt \\
     & & \Pr(y \le Y_i < y + h) \ = \ S(y)-S(y+h)
    \end{eqnarray*}
\end{itemize}

\end{frame}


\frame{
\frametitle{Survival Function}
\begin{center}
\only<1|handout:0>{\includegraphics[height=.8\textheight]{hazardsurvival-1.pdf}}%
\only<2|handout:0>{\includegraphics[height=.8\textheight]{hazardsurvival-2.pdf}}%
\only<3|handout:1>{\includegraphics[height=.8\textheight]{hazardsurvival-3.pdf}}
\end{center}
}


\frame{

  \frametitle{Hazard Function}

  \begin{itemize}
  \item \alert{Hazard function}: Instantaneous rate of leaving a state at time
    $t$ conditional on survival up to that time
    $$ \lambda(y) \ \equiv \ \lim_{h \downarrow 0} \frac{\Pr(y \le Y_i < y + h \mid Y_i \ge y)}{h}
    \ = \ \frac{f(y)}{S(y)} $$
  \item ``Force of mortality'' --- what is the `risk' that I die at time $y$ given that I have lived up until $y$?
  \item Difficult to directly interpret, but useful for model checking, etc.
\medskip
\pause
    \item One-to-one relationship with survival function:
      $$ \lambda(y) \ = \ - \frac{d}{dy} \log S(y)
      \quad \text{and} \quad S(y) \ = \ \exp\left(-\int_0^y \lambda(t) dt \right)  $$
  % \item \alert{Cumulative hazard function}:
  %   \begin{eqnarray*}
  %     \Lambda(y) & \equiv & \int_0^y \lambda(t) dt \ = \ -\log S(y)
  %   \end{eqnarray*}
  \end{itemize}

}

\frame{
\frametitle{Hazard Function}
\begin{center}
\only<1|handout:0>{\includegraphics[height=.8\textheight]{hazardsurvival-3.pdf}}%
\only<2|handout:1>{\includegraphics[height=.8\textheight]{hazardsurvival-4.pdf}}
\end{center}
}


\frame{

  \frametitle{Quantities of Interest}

  \begin{itemize}
  \addtolength{\itemsep}{0.25\baselineskip}
  \item Shape of the survival curve
\pause
\medskip
  \item \alert{Expected} (remaining) \alert{time to event} ($=$ life expectancy at age $y$):
   $$ \mu(y) \ \equiv \ \text{E}(Y_i - y \mid Y_i > y)
   \ = \ \frac{1}{S(y)} \int_y^\infty S(t) dt $$
  \item Given that I'm alive at $y$, how much longer should I expect to live?
\medskip
\pause
\item Predicted differences in the above
\medskip
\pause
\item Causal effects on survival outcomes:
  \begin{itemize}
  \addtolength{\itemsep}{.25\baselineskip}
  \item \alert{One-shot treatment} administered at the beginning of study period \\
   --- needs conditional ignorability given observed pre-trial covariates
  \item \alert{Time-varying treatment}, possibly given in response to covariates \\
  --- needs ``sequential ignorability''
  \end{itemize}
  \end{itemize}

}

\frame{

  \frametitle{Censoring}

  \begin{itemize}

  \item Observation is \alert{right-censored} when only the lower bound of
    duration is known: $Y_i \in (c, \infty)$
\pause
\medskip
  \item The \alert{independent censoring} assumption: Censored observations
    do not systematically differ from complete observations in terms of hazard rates
  \item A sufficient condition: $Y_i \indep C_i \mid X_i$ where $C_i =$ time to censoring
  \item Either $Y_i$ or $C_i$ is actually observed

\pause
  \item Examples:
    \begin{itemize}
      \item Random attrition of study sample
      \item Study begins and ends at exogenously fixed calendar dates
      \item Study ends after fixed duration (type I censoring)
      \item Study ends after a fixed number of failures (type II censoring)
    \end{itemize}

\pause
\medskip
\item Other types of censoring (left, interval) are less common

 \end{itemize}

}


% \frame{
% \frametitle{Typical Observation Schemes for Duration Analysis}
% \begin{center}
% \includegraphics[width=.8\textwidth]{censoring-1.pdf}
% \end{center}
% Q. Which observations are right/left/interval censored? \pause \\
% A. 1 is not censored; 2--4 are {\it all censored}.

% }

% %\frame{
% %\frametitle{Typical Observation Schemes for Duration Analysis}
% %\begin{center}
% %\includegraphics[width=.8\textwidth]{pics/censoring-3.pdf}
% %\end{center}
% %Q. Which observations are right/left/interval censored? \pause \\
% %A. 1-2 are not censored; 3 is right censored; 4 is left censored.
% %
% %}

% \frame{
% \frametitle{Typical Observation Schemes for Duration Analysis}
% \begin{center}
% \includegraphics[width=.8\textwidth]{censoring-2.pdf}
% \end{center}
% Q. Which observations are right/left/interval censored? \pause \\
% A. 1 is interval censored; 2--4 are right censored.

% }


\frame{

\frametitle{Cabinet Duration Example: Censoring}
King, Alt, Burns, and Laver (1990 AJPS):
\begin{itemize}
\item $Y_i$: Duration of parliamentary cabinets, $n=314$ \pause
\begin{center}
\includegraphics[scale=0.5]{cabinet.pdf}
\end{center}
\item Notice the ``bump''?
\medskip \pause
\item Some cabinets end their lives ``naturally''
\item Others end because of constitutional interelection periods (CIEP)
\medskip \pause
\item King et al. treat CIEPs as censored observations
\item But is this ``censoring'' independent?
\end{itemize}
}


\section{Nonparametric Estimation of Survival Function}

\frame{

  \frametitle{Discrete Time Approximation}

  \begin{itemize}
  \item Time is continuous but we observe discrete time: $t_1 < t_2 < \cdots$
\pause
  \item Density function: $f(t_j) = \Pr(Y_i = t_j)$
  \item Survival function: $S(t_j) = \Pr(Y_i \alert{>} t_j) = \sum_{\{k:\ t_k > t_j\}} f(t_k)$
\pause
  \item Hazard function: $\lambda(t_j) = \Pr(Y_i = t_j \mid Y_i \alert{\ge} t_j)  =  f(t_j)\big/ S(t_{\alert{j-1}})$
  % \item Cumulative hazard function: $\Lambda(y)=\sum_{\{j:\ t_j \le y\}} \lambda(t_j)$
\pause
\medskip
  \item Key relationships:
   \begin{itemize}
    \addtolength{\itemsep}{0.5\baselineskip}
    \item $S(t_j) = \prod_{k=1}^j (1-\lambda(t_k))$
\pause
    \item $f(t_j) = S(t_{j-1}) - S(t_j) = \lambda(t_j)\prod_{k=1}^{j-1}(1-\lambda(t_k))$
    \end{itemize}
\medskip
\pause
\item Expected remaining time to event:
  \begin{eqnarray*}
  \mu(t_j) & = & \text{E}(Y_i - t_j\mid Y_i > t_j) \\
  & = & \frac{1}{S(t_j)}\sum_{k=j+1}^\infty(t_k-t_j)f(t_k)
  %\ = \ \frac{1}{S(t_{j})}\sum_{k=j}^\infty (t_{k+1} - t_{k})S(t_{k})
  \end{eqnarray*}
\end{itemize}


}

\frame{

  \frametitle{Estimating the Survival Curve Without a Model}

  \begin{itemize}
  \item Goal: Get the sense of what $S(t_j)$ looks like before introducing $X_i$
\pause
\medskip
  \item Easy if no censoring; just count \# of units failing at each $t_j$
  \item Censored observations make things a bit more complicated
\pause
\medskip
  \item Setup:
    \begin{itemize}
      \addtolength{\itemsep}{0.25\baselineskip}
      \item Observed failure times: $t_1 < t_2 < \cdots < t_J$
      \item $d_j = $ \# of units that failed at time $t_j$
      \item $m_j = $ \# of units censored at time $t_j$
      \item $r_j = \sum_{k=j}^J (d_k + m_k)$ \\
        \quad $ = $ \# of units \alert{at risk} at time $t_j$, i.e.,
        those that have neither failed nor been censored
        until right before $t_j$
   \end{itemize}
  \medskip
\pause
  \item A natural estimate for the hazard function will then be:
    $$ \hat\lambda(t_j) \ = \ \widehat{\Pr}(Y_i = t_j \mid Y_i \geq t_j) \ = \ \frac{d_j}{r_j} $$
  \item In fact, this is the MLE of $\lambda(t_j)$
  \end{itemize}


}

\frame{

  \frametitle{Kaplan-Meier Estimator}

  \begin{itemize}
  \item This leads to the \alert{Kaplan-Meier estimator}:
$$\widehat{S}(t_j) \ = \ \prod_{k=j}^J(1-\hat\lambda(t_k)) \ = \ \prod_{k=j}^J \frac{r_k-d_k}{r_k}$$

\pause
\bigskip

  \item Using the MLE derivation for $\hat\lambda(t_j)$, we obtain the Hessian-based
    estimate of the asymptotic variance:

    $$\widehat{\text{Var}(\widehat{S}(t_j))} \ = \ \widehat{S}^2(t_j)\sum_{k=j}^J\frac{d_k}{r_k(r_k-d_k)}$$

  \end{itemize}

}

\frame{

  \frametitle{Example: Time Until Commitment to IMF Article VIII}

  \begin{center}
    \includegraphics[scale=0.6]{Kaplan-Meier.pdf}
  \end{center}
  Simmons (2000 APSR)
}

\section{Parametric Regression Models}

\frame{

\frametitle{Exponential Regression Model}

\begin{itemize}
\item Suppose that failures occur according to a Poisson process \\
(i.e. continuously, independently, and with constant probability)
\item Then the ``time to an event'' follows the \alert{exponential} distribution
\medskip
\pause
\item Model: $Y_i\mid X_i \sim_{\text{ind}} \text{Exponential}(\mu_i)$ where $\mu_i = \exp(X_i^{'} \beta)$

\item Density: $f(y \mid \mu_i) = \frac{1}{\mu_i} \exp(-y/\mu_i)$

\item Mean $\text{E}(Y_i \mid \mu_i) =\mu_i$ and Variance $\text{Var}(Y_i \mid \mu_i) = \mu_i^2$

\item Survival function: $S(y) = \exp(-y/\mu_i)$

\item Hazard function: $\lambda(y) = 1/\mu_i = \exp(-X_i^{'}\beta)$ \ (constant in $y$)

\pause
\medskip

\item A common alternative parameterization: $\gamma_i \equiv 1/\mu_i$
\item This changes nothing except that the sign of $\beta$ gets reversed

\pause
\medskip

\item The \alert{constant hazard} assumption: $\lambda_i$ does not vary across time
\end{itemize}

}

\frame{

\frametitle{MLE for the Exponential Model with Censoring}

\begin{itemize}
\item Censoring indicator: $D_i = 1$ if censored
\item $Y_i$ is the censoring time (rather than failure time) if $D_i = 1$
\medskip
\pause
\item Likelihood function:
  \begin{eqnarray*}
    L_n(\beta \mid Y, X, D)
    & = & \prod_{i=1}^n \underbrace{\left\{f(Y_i\mid \mu_i)\right\}^{1-D_i}}_{uncensored} \cdot
        \underbrace{\left\{S(Y_i\mid \mu_i)\right\}^{D_i}}_{censored} \\
    & = & \prod_{i=1}^n \left\{\frac{1}{\mu_i}\exp(-Y_i/\mu_i)\right\}^{1-D_i}
        \left\{\exp(-Y_i/\mu_i)\right\}^{D_i} \\
    & = & \prod_{i=1}^n \exp\left\{-(1-D_i)X_i^\top\beta\right\} \exp\left\{-\exp(-X_i^{'}\beta)Y_i\right\}
  \end{eqnarray*}
\pause
\item Log-likelihood, score and Hessian can be calculated as usual
\end{itemize}

}


\frame{

\frametitle{Weibull Regression Model}

\begin{itemize}
\item The constant hazard assumption is often too restrictive
\item The \alert{Weibull} model relaxes the assumption by introducing a ``shape'' parameter

\pause
\medskip

\item Model: $Y_i\mid X_i \sim_{\text{ind}} \text{Weibull}(\mu_i, \alpha)$ where $\mu_i = \exp(X_i^\top\beta)$ and $\alpha > 0$
\item Density: $f(y\mid \mu_i,\alpha) = \frac{\alpha}{\mu_i^\alpha}y^{\alpha-1}\exp\{-(y/\mu_i)^\alpha\}$
\item Reduces to the exponential model when $\alpha = 1$
\item Survival function: $S(y) = \exp\{-(y/\mu_i)^\alpha\}$
\item Hazard function: $\lambda(y) \ = \ \frac{\alpha}{\mu_i^\alpha}y^{\alpha-1}$

\pause
\medskip

\item The \alert{monotonic hazard} assumption:
  increasing (decreasing) if $\alpha > 1$ (if $\alpha < 1$)
\medskip
\item Other parametric regression models:
    \begin{itemize}
    \item Gompertz: Monotonic hazard
    \item Log-normal, Log-logistic: Inverse U-shaped hazard
    \end{itemize}
\end{itemize}

}

\frame{
 \frametitle{Cabinet Duration Example: Exponential or Weibull?}
 King et al. (Exponential) vs. Warwick and Easton (Weibull)
 \begin{itemize}
 \item Comparing density functions:
 \end{itemize}
 \vspace{-\baselineskip}
 \begin{center}
  \includegraphics[width=.85\textwidth]{cabinet-density.pdf}
\end{center}

}

\frame{
 \frametitle{Cabinet Duration Example: Exponential or Weibull?}
 King et al. (Exponential) vs. Warwick and Easton (Weibull)
 \begin{itemize}
 \item Comparing hazard functions:
 \end{itemize}
 \vspace{-\baselineskip}
 \begin{center}
 \includegraphics[width=.85\textwidth]{cabinet-hazard.pdf}
 \end{center}
}

\section{Cox Proportional-Hazard Model}

\frame{

  \frametitle{Semi-Parametric Regression for Survival Data}

  \begin{itemize}
  \item Less restriction on the hazard function
  \item Time-varying covariates to further model stochastic risks

\pause
\medskip
  \item Note that both exponential and Weibull models are \alert{proportional hazard models}:
\vspace{-.08in}
  $$ \lambda(y\mid X_i) \ = \ \underbrace{\lambda_0(y)}_{\text{baseline\ hazard}} \exp(X_i^{'}\beta^\ast) $$
  where $\lambda_0(y) \ = \ {\scriptsize \begin{cases} 1 & \text{(exponential)} \\ \alpha y^{\alpha-1} & \text{(Weibull)} \end{cases}}$ \ and \ $\beta^\ast = -\alpha\beta$

\pause
\bigskip
  \item The \alert{Cox Proportional Hazard Model} generalizes this model:

    $$\lambda(y\mid X_{i}(y)) \ = \ \lambda_0(y) \exp(X_i(y)^{'} \beta^\ast)$$
    where
    \begin{itemize}
    \item $\lambda_0(y)$: Nonparametric baseline hazard common to all $i$ across $t$
    \item $X_i(y)$: (Potentially) time-varying covariates
    \end{itemize}
    {\footnotesize (Note: We omit $\ast$ from hereon)}

  \end{itemize}

}


\begin{frame}
\frametitle{Example: Hazards Accommodated by the Cox Model}
\begin{center}
\includegraphics[width=.3\textwidth]{CoxHazard1.pdf}%
\includegraphics[width=.3\textwidth]{CoxHazard2.pdf}%
\includegraphics[width=.3\textwidth]{CoxHazard3.pdf}
\end{center}
\begin{itemize}
\item The Cox PH model allows flexible shapes of hazard functions
\medskip
\item Suppose we have one binary predictor $x \in \{0,1\}$ to model $y$:
\smallskip
\begin{enumerate}
\addtolength{\itemsep}{.25\baselineskip}
\item $\lambda(y\mid x) \ = \ \lambda_0(y) \exp(x\beta)$ \hfill ------ no time-varying covariate
\item $\lambda(y\mid x) \ = \ \lambda_0(y) \exp\left[x\beta_1 + xy\beta_2\right]$ \hfill ------ interaction with time trend
\item $\lambda(y\mid x) \ = \ \lambda_0(y) \exp\left[x \beta_1 + x(1.5-y)\boldsymbol{1}\{y\leq 1.5\}\beta_2\right]$ \\
      \flushright ------ allowing high initial risk
\end{enumerate}
\medskip
{\footnotesize
\item[] Note: In the figures, the \alert{relative risk} (RR) stands for:
$$ RR \ = \  \frac{\lambda(y\mid x=1)}{\lambda(y\mid x=0)}
      \ = \ \exp\left[g(y\mid x=1) - g(y\mid x=0)\right]$$
}
\end{itemize}
\end{frame}


\begin{frame}
\frametitle{Estimation of Cox Proportional Hazard Models}
\small
\vspace{-.5\baselineskip}
\begin{itemize}
  \item Joint MLE for $\lambda_0(y)$ and $\beta$ is difficult (because $\lambda_0(y)$ is nonparametric)
  \item Instead, consider the \alert{partial likelihood function} which only contains
    information about non-censored observations
\pause
  \item Because of the independent censoring assumption, this should give us
    a consistent (although not efficient) estimate for $\beta$
\pause
\smallskip
  \item For now, suppose that no two observations fail at the same time
  \item This implies we can unambiguously index observations by $j$
\pause
\medskip
  \item Under this assumption, the partial likelihood function turns out to be:
  {\footnotesize
    \begin{eqnarray*}
   L_P(\beta) & = & \prod_{j=1}^J \frac{\exp(X_i(t_j)^\top\beta)}
             {\sum_{k\in R(t_j)}\exp(X_{k}(t_j)^\top\beta)}
    \end{eqnarray*}
    where $R(t_j) = \text{risk set at time } t_j$
    }
\pause
%\vspace{-.5\baselineskip}
  \item Note that $\lambda_0(y)$ drops out of the partial likelihood
\pause
\smallskip
  \item Take the log and maximize to obtain the estimate of $\beta$
\end{itemize}
\end{frame}


\begin{frame}


\huge Hypothesis tests, Model checking, and Likelihood models!

\end{frame}



\end{document}



\begin{frame}
\frametitle{Theory of Partial Likelihood Estimation}
\small
\begin{itemize}
\item More generally, suppose outcome of interest $Y = (Y_1, Y_2, ..., Y_J)$ can be
transformed into two random variables $W$ and $V$:
    \begin{itemize}
    \item $W = (W_1, W_2, ..., W_J)$: ``Main'' component (e.g. failure time)
    \item $V = (V_1, V_2, ..., V_J)$: ``Nuisance'' component (e.g. censoring time)
    \end{itemize} \pause \medskip
\item Likelihood function for $Y$:
\begin{eqnarray*}
L(\theta; Y) & = & f(w_J, v_J, w_{J-1}, v_{J-1}, ... \mid \theta) \\
    & = & \textstyle\prod_{j=1}^J f(w_j, v_j \mid w_{j-1}, v_{j-1}, ..., \theta) \\ \pause
    & = & \underbrace{\textstyle\prod_{j=1}^J f(w_j \mid v_j, w_{j-1}, v_{j-1}, ..., \theta)}_{\equiv \ L_P(\theta): \ \text{\alert{Partial Likelihood}}} \cdot f(v_j\mid w_{j-1}, v_{j-1}, ..., \theta)
\end{eqnarray*}

\pause
\item Cox (1975): Under fairly general assumptions about the distribution of $W$ and $V$ (e.g. $W \indep V \mid \theta$), maximizing $L_P(\theta)$ leads to a valid estimate of $\theta$

\item $\hat\theta_{MPLE}$: Consistent, asymp. normal, information equality, valid Wald and (partial) LR tests, but less efficient than full MLE

\item Conditional likelihood is a special case of partial likelihood

\end{itemize}
\end{frame}






\begin{frame}
\frametitle{Partial Likelihood Function for the Cox PH Model}
\small
\begin{itemize}
\item For survival outcome $Y_1, ..., Y_n$ with discrete observed failure times:
\begin{center}
\begin{tabular}{l|c|c|c|c}
\hline
Observed failure time                 & $t_1$ & $t_2$ & $\cdots$ & $t_J$ \\
Failures at $t_j$                     & $W_1$ & $W_2$ & $\cdots$ & $W_J$ \\
Censoring between $t_{j-1}$ and $t_j$ & $V_1$ & $V_2$ & $\cdots$ & $V_J$  \\
\hline
\end{tabular}
\end{center}
\pause
\item Partial likelihood function:
\begin{eqnarray*}
L_P(\beta) & = & \prod\nolimits_{j=1}^J f(w_j \mid \underbrace{v_j, w_{j-1}, v_{j-1}, ...}_{{ = \ \Omega\setminus R(t_j) \ (\text{complement of the risk set!})}}, X, \beta)
\end{eqnarray*}
\pause
For each $t_j$, this expression asks: ``What is the probability of this particular observation $j$ failing, given that these observations ($\Omega\setminus R(t_j)$) have already exited the risk set?'' \pause Therefore,
\end{itemize}
\begin{eqnarray*}
L_P(\beta) & = & \prod\nolimits_{j=1}^J \frac{\Pr(Y_j = t_j\mid Y_j \geq t_j, X, \beta)}
               {\sum_{k\in R(t_j)} \Pr(Y_k = t_j \mid Y_k \geq t_j, X, \beta)} \\
             & = & \prod\nolimits_{j=1}^J \frac{\lambda(t_j\mid X_i(t_j),\beta)}
             {\sum_{k\in R(t_j)} \lambda(t_j\mid X_{k}(t_j),\beta)}
             =  \prod\nolimits_{j=1}^J \frac{\exp(X_i(t_j)^{'}\beta)}
             {\sum_{k\in R(t_j)}\exp(X_{k}(t_j)^\top\beta)} \qed
\end{eqnarray*}

\end{frame}


\begin{frame}
\frametitle{Dealing with Ties}
\small
\vspace{-.5\baselineskip}
\begin{itemize}
\item With discrete time data, there usually exist tied observations
\item How do we incorporate ties into the MPLE framework?
\pause
\smallskip
\item Basic idea: Break the ties in all possible ways, and average the partial likelihood
\end{itemize}
\pause
With $d_j$ observations failing at time $j$, we can calculate $L_P$ by:
\begin{enumerate}
\item ``\alert{Exact}'' method (computationally intensive):
$$   L_P(\beta)
             \ \approx \ \prod_{j=1}^J \frac{\exp\left(\sum_{i=1}^{d_j} X_i(t_j)^\top\beta\right)}
             {\sum_{S'\in S_j}\exp\left(\sum_{i'\in S'} X_{i'}(t_j)^\top\beta \right)}
$$
where $S_j$ is the set of all possible combinations of $d_j$ units from $R(t_j)$
\pause
\item \alert{Breslow-Pito} approximation:
    $$ L_P(\beta) \ \approx \ \prod_{j=1}^J
    \frac{\exp\left(\sum_{i=1}^{d_j} X_{i}(t_j)^\top\beta\right)}
    { \left\{\sum_{i'\in R(t_j)} \exp\left(X_{i'}(t_j)^\top\beta\right)\right\}^{d_j}} $$
\pause
\vspace{-.75\baselineskip}
\item Efron approximation: Similar to Breslow-Pito, with some adjustment
\end{enumerate}
\end{frame}


\begin{frame}
\frametitle{Estimating the Baseline Hazard Function}
\begin{itemize}
  \item MPLE gives $\hat\beta$, but what about the baseline hazard, $\lambda_0(y)$?
\pause
  \item Basic idea: Apply the Kaplan-Meier framework conditional on $\hat\beta$
\pause
  \item Derivation is complicated, so we omit details here
\end{itemize}
\pause
Warwick (1992 {\it AJPS}) applies Cox PH to cabinet duration data:
    \begin{center}
      \includegraphics[width=0.5\textwidth]{hazardA.pdf}%
      \includegraphics[width=0.5\textwidth]{hazardF.pdf}%
    \end{center}
\begin{itemize}
\item Baseline hazard appears to vary by countries
\end{itemize}
\end{frame}


\section{Competing Risks Models}

\frame{

  \frametitle{Allowing Multiple Failure Types}
\small
  \begin{itemize}
    \addtolength{\itemsep}{0.5\baselineskip}
  \item Observations may terminate for multiple different reasons \\
    (e.g. patients die of different diseases)
  \item And covariates may affect those failure types differently
\pause
\medskip
  \item Suppose there are $K$ failure types, or ``causes'' of failure
  \item For each $k\in\{1,...,K\}$ we can define a
    \alert{cause-specific hazard function}:
  $$\lambda_k(y) \ = \ \lim_{h \downarrow 0}
    \frac{\Pr(y \le Y_i < y + h, F_i = k \mid Y_i \ge y)}{h}$$
    where $F_i$ denotes the cause of failure
\pause
    \item The \alert{total hazard function}:
  $$\lambda(y) \ = \ \sum_{k=1}^K \lambda_k(y)$$
    This is the instantaneous risk of failure due to any of the $K$ causes

 \end{itemize}

}


\frame{

  \frametitle{Quantities of Interest for the Competing Risks Model}
\small
  \begin{itemize}
    \item The \alert{overall survival function}:
    $$S(y) \ = \ \exp\left(- \int_0^y \lambda(t) dt \right) \ = \ \prod_{k=1}^K \exp\left(-\int_0^y \lambda_k(t) dt  \right)$$
    This is the probability of being free from any failure type until time $y$
\pause
\bigskip
  \item The \alert{cumulative incidence function}:
    $$I_k(y) \ \equiv \ \Pr(Y_i \le y, F_i = k) \ = \ \int_0^y \lambda_k(t)S(t)dt$$
    This is the probability of failure due to cause $k$ before time $y$
\end{itemize}

}

\frame{

\frametitle{Cumulative Incidence Function Example}

\includegraphics[scale=0.8]{CIF.pdf}

}


\frame{

\frametitle{Identifiability of Latent Failure Times}
\begin{itemize}
  \item \alert{Latent failure times}: $Y_i \ = \ \min\left\{Y_{i1}^\ast, ..., Y_{iK}\right\}$ \medskip \pause
  \item Ideally, we would be interested in a \alert{joint survival function}
  (e.g. How long would a cabinet last if there were no CIEP failures?):
  $$ S^\ast(y_1, ..., y_K) \ = \ \Pr(Y_{i1}^\ast > y_1, ..., Y_{iK}^\ast > y_K) $$
  \item However, this quantity is \alert{unidentifiable} without strong assumptions (e.g.
  causes are jointly independent of one another) \pause \medskip

  \item Can \alert{cause-specific (marginal) survival function} be identified?
  $$ S^\ast_k(y) \ = \ \Pr(Y_{ik}^\ast > y)$$

  \pause

  \item What about the cause-specific hazard function? Overall survival function? \pause \medskip

  \item In general, quantities involving latent failure times are unidentified
  \item The problem is analogous to the fundamental problem of causal inference
  (i.e. only one failure type occurs to each unit)
\end{itemize}
}

\frame{

  \frametitle{Estimation of the Competing Risks Model}
\small
Without covariates:
  \begin{itemize}
    \addtolength{\itemsep}{-.25\baselineskip}
    \item Cause-specific hazards can be estimated by Kaplan-Meier:
      $$\hat\lambda_k(t_j) = d_{jk} \big/ r_j$$
      where $d_{jk} =$ \# failing due to cause $k$ at time $t_j$
\pause
    \item Plug this in to estimate the survival and cumulative incidence functions:
      $$ \widehat{S}(t_j) = \prod\nolimits_{l \leq j} \left\{1 - \sum\nolimits_{k=1}^K \hat\lambda_k(t_l)\right\} \quad \text{and} \quad \widehat{I}_k(t_j) \ = \ \sum\nolimits_{j: t_l \le t_j} \hat{\lambda}_k(t_l) \widehat{S}(t_{l-1})$$
    \end{itemize}

\pause
\medskip
With covariates:
  \begin{itemize}
    \addtolength{\itemsep}{-.25\baselineskip}
    \item The cause-specific hazards can also be modelled as functions of $X_i$
    \item The likelihood function (assuming no ties):
      \vspace{-.05in}
      \begin{eqnarray*}
       L(\beta) & = & \prod\nolimits_{j=1}^J \left\{ f_{F_j}(t_j) \prod\nolimits_{k \neq F_j} S_k(t_j)\right\}
    \ = \ \prod\nolimits_{j=1}^J \left\{ \lambda_{F_j}(t_j) \prod\nolimits_{k=1}^K S_k(t_j)\right\} \\
    & = & \prod\nolimits_{j=1}^J \left\{ \lambda_{F_j}(t_j) \prod\nolimits_{k=1}^K \prod\nolimits_{l=1}^j (1-\lambda_{k}(t_l))\right\}
    \end{eqnarray*}
\pause
\vspace{-.1in}
   \item Use Cox PH for $\lambda_k(t_j)$ and estimate by MPLE
  \end{itemize}
}



\frame{

  \frametitle{Cabinet Duration Example: Competing Risks}
\small
Diermeier and Stevenson (1999 {\it AJPS})
  \begin{itemize}
 \item Deal with the censoring problem by distinguishing two failure types:
   \begin{enumerate}
     \addtolength{\itemsep}{0.25\baselineskip}
     \item Dissolution: Calls for a new election and resets the clock
     \item Replacement: The new cabinet still faces the same mandatory election
   \end{enumerate}
\vspace{-.15in}
\begin{center}
  \includegraphics[height=0.65\textheight]{cabinet-comprisks.pdf}
\end{center}

 \end{itemize}

}



\section{Conclusion}

\frame{
 \frametitle{Concluding Remarks}
 \begin{itemize}
 \item Many examples of survival/duration data in political science
 \item Close connection with causal inference literature
 \item Once again, report quantities of interest, not just coefficients
  \end{itemize}
 \medskip
Further generalizations we couldn't cover:
\medskip
 \begin{columns}[T]
 \column{.55\textwidth}
  \begin{enumerate}
 \item Time-varying coefficients
 \item Group-varying baseline hazards
 \item Recurrent events
 \item Correlated failure time
 \item Multistate models
 \end{enumerate}
 \column{.4\textwidth}
 \includegraphics[width=\textwidth]{multistate.pdf}
 \end{columns}
}


\end{document}
